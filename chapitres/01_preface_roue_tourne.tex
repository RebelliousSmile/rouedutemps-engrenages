% Fichier: chapitres/01_preface_roue_tourne.tex
% Chapitre 1: Préface - La Roue Tourne
% Transcrit depuis: first-draft/chapitre01_preface_roue_tourne.md
% Date transcription: 2025-11-08
% Univers: La Roue du Temps

\section{Préface -- La Roue Tourne}

Ce document absolument non-officiel est ouvert à toutes les bonnes volontés (n'hésitez pas à me contacter sur fx.rebellious.smile@gmail.com si vous souhaitez participer). Il a pour objectif d'adapter le système d'Engrenages (d'Alexandre Clavel) à l'univers de La Roue du Temps.

La Roue du Temps est un récit de dizaines de milliers de pages écrit par Robert Jordan, puis terminé par Brandon Sanderson. Les livres ont été plusieurs fois traduit en français, chez plusieurs éditeurs, notamment chez Bragelonne. L'œuvre est devenu aujourd'hui à nouveau dans La Lumière grâce à son adaptation en série TV sur Amazon, sur laquelle cette adaptation s'appuie, car toutes les chances sont pour que si vous vous intéressez à cette saga aujourd'hui, c'est grâce au petit écran.

Un jeu de rôles intitulé \textit{Wheel of Time} a été publié pour le d20 au début du siècle. Une campagne nommée ``Les prophéties du Dragon'' a également été publiée. Une version remaniée sera proposée plus tard en complément de cette adaptation.

Je vous invite à consulter les excellents wiki de La Roue du Temps, en version originale ou en version française : \url{https://encyclo.rouedutemps.fr} si vous êtes curieux du monde dans lequel vous allez vivre des aventures.

Je vous souhaite une bonne lecture et j'espère vous donner envie d'arpenter les Terres de L'Ouest en utilisant les règles d'Engrenages.

\subsection{Que joue-t-on ?}

\begin{quotebox}
  La Roue du Temps tourne, et les Âges viennent et passent, laissant des souvenirs qui deviennent des légendes. Puis même les légendes s'estompent pour devenir des mythes, et sont oubliées quand revient l'Âge qui leur a donné naissance.
\end{quotebox}

La Roue du Temps mélange les Parques et le concept de la réincarnation, tourne et crée des âges, 7 si l'on en croit le nombre de rayons sur les illustrations. Le premier âge correspond à notre monde. Le Second Âge a permis des merveilles de magie jusqu'à ce qu'une \textit{Aes Sedai} libère Le Ténébreux en croyant toucher une nouvelle forme de magie débarrassée des genres. Une guerre absolue entre le ``Dragon'' et les forces du Ténébreux a été remportée par La Lumière, mais au moment de sceller la prison magique, Le Ténébreux a eu le temps de contre-attaquer et de corrompre la source de magie des hommes. Ceux-ci sont devenus fous et ont ravagé le monde avec leur magie. Le Troisième Âge part de ce monde en ruines pour reconstruire de belles nations. Il est dit que cet Âge prendra fin avec l'Ultime Bataille, qui opposera à nouveau le Dragon aux forces des Ténèbres.

Notez que le combat entre Le Créateur et Le Ténébreux ne s'aligne pas sur les notions de bien et de mal. Pour stopper la progression de leurs ennemis, les combattants de La Lumière doivent faire bien des sacrifices, y compris de mauvaises actions ; d'ailleurs, peu de ces combattants mettent en avant leur vertu. À commencer par le premier d'entre eux, le Dragon, dont la précédente incarnation était appelée avec raison Meurtrier des siens (\textit{Kinslayer}).

Dans ce contexte, les joueurs vont incarner des \textit{ta'veren}, des acteurs particuliers de La Roue du Temps qui bénéficient, fréquemment à leur insu, d'une plus grande chance de faire changer la grande Histoire : ce dernier leur offre plus d'opportunités de devenir célèbre et d'affecter et d'influencer le monde. Et pour celles qui tissent des fils avec Le Pouvoir de l'Unique et qui perçoivent la trame du monde, les \textit{ta'veren} apparaissent comme des points névralgiques de La Roue, où les fils du Dessin se concentrent.

Le récit de Robert Jordan, inspiré de celui de Tolkien, se concentre sur un berger, un jeune homme sans histoire qui va se retrouver avec l'issue du conflit majeur du monde sur ses épaules. Cet homme, futur ``Dragon'', est donc épaulé par ses proches, ses amis, lesquels marchent avec lui et se faisant bénéficient eux aussi d'une destinée extraordinaire.

L'ambiance de La Roue du Temps inclut ces éléments :

\begin{itemize}
  \item Les cités sont cosmopolites, et les campagnes dangereuses. Plus il y a de monde, et plus des témoins vont répéter ce qu'ils ont vu, et qui ils ont vu.
  \item Chaque nation a de fortes traditions, et les exporte avec lui, de sorte qu'on reconnaît rapidement d'où vient une personne à ses habits, ses expressions, ses habitudes, ses comportements.
  \item Les \textit{Aes Sedai} pensent qu'elles sauvent le monde, et les autres pensent qu'elles manipulent les gens à des fins personnelles. Leur sororité reste le principal ascenseur social pour les femmes.
  \item Le combat pour survivre est omniprésent. Seul ceux qui suivent le Paradigme de la Feuille sont pacifistes et refusent de combattre, les gens normaux sont habitués à devoir se défendre et se battre pour prendre ou garder leur place.
  \item La magie a réussi par le passé des œuvres extraordinaires, mais elle est aussi la cause des grandes destructions civilisationnelles. Le commun s'en méfie comme la peste.
  \item L'habit ne fait pas le moine. Les Fils de La Lumière sont des inquisiteurs tortionnaires qui utilisent leur croyance pour justifier toutes leurs actions, et bien des gens respectueux peuvent être des Suppôts des Ténèbres. Mais les héros peuvent ramener les âmes égarées dans le droit chemin.
  \item \textit{Le devoir est plus lourd qu'une montagne, la mort plus légère qu'une plume.} L'omniprésence des batailles et les prophéties apocalyptiques rendent les gens fatalistes et les sociétés articulées autour du devoir. Car le combat souffre rarement de fantaisie.
\end{itemize}

\subsection{Quand joue-t-on ?}

Les adaptations d'une œuvre en jeu de rôles se heurtent toujours au fait que la trame principale a déjà été écrite, et autour des protagonistes autres que les PJs. La tâche délicate du maître de jeu est de trouver une proposition dans laquelle les joueuses et les joueurs garderont l'excitation de jouer dans un univers qui leur plait, et peut-être même avoir le plaisir de croiser des PNJs de la série ou des romans, mais sans que ces derniers puissent leur faire de l'ombre. Et sauf à jouer des \textit{What If}, le rôle du Dragon Réincarné est déjà attribué.

Voici quelques propositions pour trouver votre angle d'attaque.

\begin{itemize}
  \item Le premier axe sur lequel vous pouvez jouer est l'Histoire. Jouez à la fin du Second Âge, à l'époque de Lews Therin \textit{Kinslayer}, ou à l'époque d'Artur Aile-de-Faucon, ou encore après le final de la série (que je n'évoquerais pas pour éviter de divulgacher).
  \item Le deuxième axe sur lequel vous pouvez jouer est la Géographie. Une campagne côté Seanchan, à l'époque du cycle, laisse tout un continent comme terrain de jeu, et peut même lorgner sur une uchronie si l'incursion des Seanchans dans les Terres de L'Ouest devait avoir plus d'impact que dans la série. Le pays de Shara, énorme surface, ne demande également qu'à vivre sous l'action de vos PJs. Enfin, dans un style plus survivaliste, on pourrait imaginer des histoires au cœur de la Flétrissure, dans une ambiance plus sombre que celle proposée ici.
  \item La campagne officielle ``Les prophéties du Dragon'' prend le parti d'entremêler ses intrigues avec l'histoire des \textit{ta'veren} des Deux-Rivières. Une perspective très excitante pour les fans, mais qui peut être limitante sur le très long terme, car les PJs finiront bien par entrer en concurrence avec la destinée des héros de Robert Jordan. Là aussi, si l'on accepte de partir sur un \textit{what if}, les PJs peuvent supplanter la version officielle et écrire leur propre épopée, loin de Rand Al'Thor et de ses amis. Et si Rand Al'Thor n'était qu'un faux Dragon ?
  \item Pour des joueuses et des joueurs novices, au cours d'un one shot ou d'une mini-campagne, jouer les héros de la série est une option à ne pas négliger. Parfois, il est plus rassurant de marcher dans les traces de quelqu'un d'autre.
\end{itemize}

% Fin du chapitre 1
% Lignes source: 1-43
% Corrections appliquées: Majuscules conceptuelles (La Roue, Le Ténébreux, etc.), italiques pour termes Vieille Langue (ta'veren, Aes Sedai, Kinslayer), guillemets LaTeX
