\section{Adapter les Règles}

\subsection{Système de Dés Typés : Lumière et Ténèbres}

\textbf{Composition des dés :}
\begin{itemize}
  \item \textbf{1 dé de Lumière} (d6 blanc ou clair)
  \item \textbf{1 dé de Ténèbres} (d6 noir ou sombre)
  \item \textbf{Dés supplémentaires} (confrontation, prise de risques) : dés neutres (couleur standard)
\end{itemize}

\subsubsection{Effets du Typage}

\textbf{Sur le dé de Lumière :}
\begin{itemize}
  \item \textbf{Résultat de 6} : Génère 1 point de \textbf{Destinée}
  \item Symbolise l'influence bénéfique du Créateur
  \item Favorise les événements positifs et les coïncidences heureuses
\end{itemize}

\textbf{Sur le dé de Ténèbres :}
\begin{itemize}
  \item \textbf{Résultat de 1} : Génère 1 point de \textbf{Stress}
  \item Symbolise l'influence corruptrice du Ténébreux
  \item Entraîne des complications et des événements défavorables
\end{itemize}

\textit{Exemple pratique :} Un test de Combat (5) + Traits (2) + Difficulté (0) avec résultats [Lumière : 6] [Ténèbres : 3] donne un total de 16, soit Marge +6, et génère 1 point de Destinée.

\subsection{Compétences Modifiées}

\textbf{Combat unifié :} Une seule compétence Combat remplace Pugilat, Escrime et Tir.

\subsubsection{Nouvelles Compétences Physiques}

\paragraph{Combat}
Regroupe Pugilat, Escrime et Tir. Les particularités de chaque combattant ressortent dans les spécialisations qu'ils vont prendre. Un maître qui possède une épée marquée du héron aura une spécialisation bienvenue à l'épée. Les Aiels auront une spécialisation Lance. Les officiers auront une spécialisation en Tactique ou en Stratégie.

\textbf{Spécialisations disponibles :}
\begin{itemize}
  \item \textbf{Épée} : Arme noble par excellence, symbole de statut
  \item \textbf{Lance} : Arme privilégiée des Aiels et cavaliers
  \item \textbf{Arc} : Arme de chasse et guerre à distance
  \item \textbf{Tactique} : Commandement et stratégie de groupe
  \item \textbf{Garde du corps} : Protection rapprochée de personnalités
  \item \textbf{Stratégie} : Planification militaire à grande échelle
\end{itemize}

\textit{Application technique :} Un personnage avec Combat 5 (Épée) bénéficie d'un bonus de +2 lors de l'utilisation d'armes blanches spécialement, mais n'a aucun bonus particulier avec d'autres armes.

\paragraph{Animaux}
Compétence qui permet de comprendre les animaux, de les monter, de les apprivoiser.

\textbf{Spécialisations :}
\begin{itemize}
  \item \textbf{Équitation} : Essentielle pour les cultures nomades et militaires
  \item \textbf{Frères des Loups} : Communication avec les loups (trait très rare)
  \item \textbf{Trollocs} : Connaissance des créatures hybrides de L'Ombre
  \item \textbf{Chasse} : Pistage et capture d'animaux sauvages
  \item \textbf{Dressage} : Entraînement et domestication
\end{itemize}

\paragraph{Mer}
Concerne tout ce qui est lié aux fleuves et aux bateaux.

\textbf{Spécialisations :}
\begin{itemize}
  \item \textbf{Chantiers navals} : Construction et réparation navale
  \item \textbf{Pleine mer} : Navigation, Natation
  \item \textbf{Voiles} : Manœuvre optimale des navires à voiles
  \item \textbf{Canotage} : Embarcations légères et navigation fluviale
  \item \textbf{Nœuds} : Matelotage et cordages
\end{itemize}

\paragraph{Dépassement}
Compétence un peu particulière, qui intervient quand vous utiliser une compétence de manière continue (enchaîner les combats dans un combat de masse), puiser dans ses ressources pour un long rituel ou pour une fuite effrénée, ou encore tenter des actions qu'on a encore jamais faites.

\textbf{Spécialisations :}
\begin{itemize}
  \item \textbf{Fatigue} : Résistance à l'épuisement physique
  \item \textbf{Exploits} : Accomplir des prouesses au-delà de ses limites
  \item \textbf{Rituels} : Maintien de la concentration pour les longs tissages
  \item \textbf{Mêlée} : Combat prolongé contre multiple adversaires
  \item \textbf{Environnement hostile} : Survie dans le désert, glaces, monde minéral
  \item \textbf{Encaissement} : Résistance à la douleur et aux blessures
\end{itemize}

\subsubsection{Nouvelles Compétences Mentales}

\paragraph{Érudition}
Correspond à la compétence Sciences, mais le monde de La Roue du Temps n'a guère de scientifiques.

\textbf{Domaines d'expertise :}
\begin{itemize}
  \item \textbf{Histoire} : Connaissance des Âges et événements passés
  \item \textbf{Géographie} : Cartographie et connaissance des nations
  \item \textbf{Langues} : Nombre de langues = Érudition $\div$ 2 + langue maternelle
  \item \textbf{Généalogies} : Lignées nobles et alliances politiques
\end{itemize}

\textit{Exemple d'application :} Érudition 6 permet de parler 4 langues (3 + langue maternelle) et d'identifier la plupart des artefacts de l'Âge des Légendes.

\paragraph{Folklore}
Concerne les savoirs basées sur la mémoire, la tradition écrite et la tradition orale.

\textbf{Spécialisations :}
\begin{itemize}
  \item \textbf{Prophéties} : Interprétation des textes prophétiques
  \item \textbf{Histoire du Pouvoir de l'Unique} : Légendes sur l'usage de la magie
  \item \textbf{Histoire} : Événements passés et chronologies
  \item \textbf{Récits fondateurs des nations et des peuples} : Mythes de création et traditions
\end{itemize}

\paragraph{Thaumatologie}
Étude théorique de la magie et des phénomènes surnaturels.

\textbf{Applications pratiques :}
\begin{itemize}
  \item Identification des \textit{ter'angreal} et \textit{angreal}
  \item Compréhension des manifestations du Pouvoir Unique
  \item Analyse des distorsions temporelles et spatiales
\end{itemize}

\subsubsection{Modificateurs Sociaux Ajoutés}

\textbf{Rang social} : +1 à +3 selon le statut

\textbf{Réputation} : +2 à -2 selon la renommée

\textbf{Différence culturelle majeure} : -2 à -4

\subsubsection{Compétences Occultes : Les Cinq Pouvoirs}

Les compétences occultes représentent la maîtrise des cinq éléments du Pouvoir Unique.

\paragraph{Air}
Manipulation des gaz, sons et forces invisibles.

\textbf{Applications techniques :}
\begin{itemize}
  \item \textbf{Mouvement} : Déplacement d'objets et personnes
  \item \textbf{Son} : Amplification, transmission, silence
  \item \textbf{Illusion} : Distorsion de la lumière et des perceptions
  \item \textbf{Champ de force} : Barrières protectrices invisibles
\end{itemize}

\textit{Exemple de tissage :} Créer un mur d'Air pour bloquer une charge de cavalerie nécessite Air 4+ avec spécialisation Champ de force.

\paragraph{Eau}
Contrôle des liquides et arts de guérison.

\textbf{Spécialisations essentielles :}
\begin{itemize}
  \item \textbf{Guérison} : Réparation des blessures et maladies
  \item \textbf{Purification} : Élimination des toxines et corruptions
  \item \textbf{Création d'eau} : Production de liquides vitaux
  \item \textbf{Glace} : Solidification et sculpture de l'eau gelée
\end{itemize}

\textit{Restriction importante :} La guérison ne peut cibler directement le canalisateur lui-même.

\paragraph{Terre}
Manipulation de la matière solide et des structures.

\textbf{Applications pratiques :}
\begin{itemize}
  \item \textbf{Sculpture} : Modification instantanée de la pierre et du métal
  \item \textbf{Séisme} : Déstabilisation du terrain et des fondations
  \item \textbf{Minéral} : Détection et extraction des métaux précieux
  \item \textbf{Squelette} : Réparation des fractures osseuses
\end{itemize}

\paragraph{Feu}
Contrôle de la chaleur et de l'énergie destructrice.

\textbf{Utilisation militaire :}
\begin{itemize}
  \item \textbf{Attaque à distance} : Boules de feu et projectiles enflammés
  \item \textbf{Arme enflammée} : Enhancement des armes conventionnelles
  \item \textbf{Chaleur} : Contrôle de la température ambiante
  \item \textbf{Combustion} : Ignition et extinction des flammes
\end{itemize}

\paragraph{Esprit}
Manipulation des consciences et liens psychiques.

\textbf{Applications sensibles :}
\begin{itemize}
  \item \textbf{Compulsion} : Influence directe sur la volonté (éthiquement problématique)
  \item \textbf{Guérison mentale} : Traitement des traumatismes psychologiques
  \item \textbf{Effacement} : Suppression sélective de souvenirs
  \item \textbf{Liens} : Création de connexions empathiques (\textit{Gaidin}/\textit{Aes Sedai})
\end{itemize}

\textit{Note éthique :} L'usage d'Esprit pour la manipulation mentale est strictement encadré par les Serments des \textit{Aes Sedai}.

\subsection{Système de Difficulté}

\begin{center}
\begin{tabular}{|l|c|p{4cm}|p{5cm}|}
\hline
\textbf{Difficulté} & \textbf{Modificateur} & \textbf{Description} & \textbf{Exemples} \\
\hline
\textbf{Trivial} & +4 & Tâches quotidiennes sans stress & Reconnaître la bannière de sa propre nation \\
\hline
\textbf{Facile} & +2 & Actions simples dans un environnement favorable & Négocier avec un marchand sur un marché familier \\
\hline
\textbf{Moyen} & 0 & Tâches exigeant concentration et compétence & Identifier un \textit{ter'angreal} mineur \\
\hline
\textbf{Difficile} & -2 & Actions complexes nécessitant expertise & Tisser un sort de guérison complexe \\
\hline
\textbf{Très difficile} & -4 & Prouesses exceptionnelles & Survivre seul dans la Grande Flétrissure \\
\hline
\textbf{Légendaire} & -6 & Accomplissements dignes des légendes & Ressusciter un mort récent avec Le Pouvoir \\
\hline
\textbf{Impossible} & -8 & Défie les lois naturelles & Voyager dans le temps \\
\hline
\end{tabular}
\end{center}

\subsubsection{Modificateurs Spécifiques}

\begin{center}
\begin{tabular}{|l|c|p{7cm}|}
\hline
\textbf{Situation} & \textbf{Modificateur} & \textbf{Notes} \\
\hline
\textbf{Proximité d'un \textit{stedding}} & -6 & Aux tests de Pouvoir uniquement \\
\hline
\textbf{Présence de L'Ombre} & -2 à -4 & Selon l'intensité de la corruption \\
\hline
\textbf{Méfiance établie} & -2 à -6 & Selon le niveau d'hostilité \\
\hline
\textbf{Terrain familier} & +2 & Connaissance du lieu \\
\hline
\textbf{Rang social approprié} & +2 & Interaction dans sa classe sociale \\
\hline
\textbf{Réputation favorable} & +1 à +3 & Selon la renommée \\
\hline
\textbf{Différence culturelle majeure} & -2 à -4 & Incompréhension mutuelle \\
\hline
\end{tabular}
\end{center}

\subsection{Interprétation des Marges}

\begin{center}
\begin{tabular}{|c|l|p{8cm}|}
\hline
\textbf{Marge} & \textbf{Qualité} & \textbf{Effets} \\
\hline
\textbf{0-1} & Réussite minimale & Objectif atteint sans élégance, conséquences possibles \\
\hline
\textbf{2-3} & Réussite satisfaisante & Accomplissement standard conforme aux attentes \\
\hline
\textbf{4-5} & Réussite remarquable & Dépassement des attentes, bénéfices collatéraux \\
\hline
\textbf{6+} & Réussite exceptionnelle & Performance mémorable, effets supplémentaires garantis \\
\hline
\end{tabular}
\end{center}

\subsection{Système de Destinée et de Stress}

\subsubsection{Acquisition des Points}

\textbf{Points de Destinée :}
\begin{itemize}
  \item Générés par les 6 sur le dé de Lumière
  \item Représentent l'influence bénéfique du Créateur
  \item Stockés dans une réserve commune accessible à tous moments
\end{itemize}

\textbf{Points de Stress :}
\begin{itemize}
  \item Générés par les 1 sur le dé de Ténèbres
  \item Symbolisent la corruption grandissante de L'Ombre
  \item S'accumulent jusqu'à atteindre un seuil critique
\end{itemize}

\subsubsection{Utilisation des Points de Destinée}

\textbf{Dépense de 1 point :}
\begin{itemize}
  \item \textbf{Relancer le dé de Lumière} : Nouvelle chance sur le dé favorable
  \item \textbf{Bonus de Marge} : +1 Marge sur le résultat final
\end{itemize}

\textbf{Dépense de tous les points :}
\begin{itemize}
  \item \textbf{Modification narrative} : Changer un détail mineur de la situation
  \item \textbf{Coïncidence favorable} : Intervention du hasard en faveur du personnage
\end{itemize}

\textit{Exemple d'usage :} Un personnage avec 3 points de Destinée peut dépenser 1 point pour relancer son dé de Lumière lors d'un test crucial, ou dépenser les 3 pour qu'un allié arrive opportunément.

\subsubsection{Gestion du Stress}

\textbf{Accumulation :}
\begin{itemize}
  \item Les points de Stress s'ajoutent automatiquement
  \item Représentent la pression croissante de L'Ombre
  \item Affectent progressivement les capacités du personnage
\end{itemize}

\subsubsection{Effets du Stress}

\begin{center}
\begin{tabular}{|c|l|l|}
\hline
\textbf{Points de Stress} & \textbf{Effets} & \textbf{Malus} \\
\hline
\textbf{1-2} & Tension nerveuse, irritabilité & Aucun \\
\hline
\textbf{3-4} & Nervosité visible & -1 aux tests sociaux \\
\hline
\textbf{5-6} & Difficultés de concentration & -2 aux tests de concentration \\
\hline
\textbf{7+} & Risque de perte de contrôle & Tests de Volonté requis \\
\hline
\end{tabular}
\end{center}

\subsubsection{Élimination du Stress}

\begin{center}
\begin{tabular}{|l|c|p{6cm}|}
\hline
\textbf{Méthode} & \textbf{Réduction} & \textbf{Conditions} \\
\hline
\textbf{Repos complet} & -1 point/jour & Détente sans contrainte \\
\hline
\textbf{Réconfort social} & -1 point & Interaction positive significative \\
\hline
\textbf{Accomplissement d'objectif} & -2 points & Réussite majeure personnelle \\
\hline
\end{tabular}
\end{center}

\subsection{Règles de Combat Adaptées}

\subsubsection{Calcul de l'Initiative}

\begin{verbatim}
Athlétisme + Combat + 1d6 + modificateurs de surprise
\end{verbatim}

\subsection{Règles Spécifiques au Pouvoir Unique}

\subsubsection{Tests de Canalisation}

\textbf{Formule de base :}
\begin{verbatim}
Somme des Pouvoirs requis + 2d6 + Traits + Difficulté + modificateurs ≥ 10
\end{verbatim}

\subsubsection{Modificateurs de Tissage}

\begin{center}
\begin{tabular}{|l|c|p{6cm}|}
\hline
\textbf{Facteur} & \textbf{Modificateur} & \textbf{Description} \\
\hline
\textbf{1 élément} & 0 & Tissage simple \\
\hline
\textbf{2 éléments} & -2 & Complexité modérée \\
\hline
\textbf{3+ éléments} & -4 & Tissage très complexe \\
\hline
\end{tabular}
\end{center}

\vspace{0.5cm}

\begin{center}
\begin{tabular}{|l|c|p{6cm}|}
\hline
\textbf{Connaissance} & \textbf{Modificateur} & \textbf{Maîtrise} \\
\hline
\textbf{Tissage sauvage} & -4 & Improvisation totale \\
\hline
\textbf{Tissage inconnu} & -2 & Première tentative \\
\hline
\textbf{Tissage étudié} & 0 & Compréhension théorique \\
\hline
\textbf{Tissage pratiqué} & +2 & Expérience régulière \\
\hline
\textbf{Tissage maîtrisé} & +4 & Expertise complète \\
\hline
\end{tabular}
\end{center}

\textit{Exemple de calcul :} Guérison complexe nécessitant Eau (6) + Esprit (3) + modificateur 2 éléments (-2) + tissage pratiqué (+2) = jet final à +9.

\subsubsection{Effets des Dés Typés sur la Magie}

\paragraph{Usage du \textit{Saidin} (Hommes)}

\textbf{Risque de la Souillure :}
\begin{itemize}
  \item Si le \textbf{dé de Ténèbres} obtient le résultat le plus élevé :
  \begin{itemize}
    \item S'ajoute à la Marge obtenue (effet amplificateur)
    \item S'ajoute également au Contrecoup (corruption progressive)
    \item Le Contrecoup n'est \textbf{pas limité} par la compétence
  \end{itemize}
\end{itemize}

\textit{Mécanisme de folie :} Accumulation progressive des Contrecoups menant à l'instabilité mentale puis à la démence destructrice.

\paragraph{Usage du \textit{Saidar} (Femmes)}

\textbf{Harmonie naturelle :}
\begin{itemize}
  \item Si le \textbf{dé de Lumière} obtient le résultat le plus élevé :
  \begin{itemize}
    \item S'ajoute à la Marge \textbf{OU} diminue le Contrecoup (choix du joueur)
    \item Représente l'harmonie naturelle des femmes avec leur moitié du Pouvoir
  \end{itemize}
\end{itemize}

\subsubsection{Collaboration Magique}

\textbf{Cercles de canalisation :}
\begin{itemize}
  \item Chaque participant ajoute \textbf{+2} au test du meneur de cercle
  \item Seul le meneur effectue le jet final
  \item La Marge est limitée par la compétence la plus faible du groupe
  \item Les Contrecoups se répartissent entre tous les participants
\end{itemize}

\textit{Application tactique :} Un cercle de 3 \textit{Aes Sedai} donne +4 au test du meneur, permettant d'accomplir des prodiges impossibles individuellement.
