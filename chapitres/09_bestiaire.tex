\chapter{Bestiaire}

\section{Règles des Créatures et Personnages}

Ce chapitre présente les adversaires et alliés que les personnages peuvent rencontrer dans l'univers de La Roue du Temps. Chaque entrée utilise le système Engrenages adapté pour faciliter leur utilisation en jeu.

\subsection{Format des Profils}

\textbf{Stats~:} Indique les modificateurs Engrenages selon le format standard des PNJ
\begin{itemize}
\item \textbf{Physique, Mental, Social}~: Modificateurs de base pour les tests d'opposition
\item \textbf{Magie}~: Modificateur pour les utilisateurs du Pouvoir Unique (le cas échéant)
\item \textbf{Jauges}~: Points de résistance de la créature (base 5 pour figurants, jusqu'à 25 pour les légendes)
\end{itemize}

\subsection{Capacités Spéciales}

Les capacités spéciales remplacent les statistiques complexes par des \textbf{effets narratifs clairs}. Elles fonctionnent selon ces principes~:

\textbf{Effets Automatiques~:} Certaines capacités s'appliquent sans test (Vol, Vision nocturne)

\textbf{Tests de Résistance~:} D'autres nécessitent un test du PJ pour résister
\begin{itemize}
\item \textbf{Format}~: Compétence + Attribut + 2d6 ≥ Seuil
\item \textbf{Exemple}~: ``Test \textbf{Faconde + Esprit ≥ 12} ou approche volontaire''
\end{itemize}

\textbf{Modificateurs Situationnels~:} Bonus/malus appliqués selon les circonstances
\begin{itemize}
\item \textbf{+X aux attaques}~: Avantage tactique de la créature
\item \textbf{-X aux tests}~: Pénalité imposée aux adversaires
\end{itemize}

\subsection{Faiblesses}

Chaque créature possède des \textbf{faiblesses exploitables} qui permettent aux PJ d'élaborer des stratégies~:

\textbf{Vulnérabilités~:} Sensibilités à certains éléments ou situations

\textbf{Limitations~:} Contraintes comportementales ou physiques

\textbf{Conditions~:} Circonstances qui neutralisent leurs avantages

\emph{Ces faiblesses ne sont jamais arbitraires mais découlent logiquement de la nature de chaque être, offrant aux joueurs astucieux des moyens de surmonter des défis apparemment insurmontables.}

\section{Les Serviteurs des Ténèbres}

Ce monde abrite des créatures qui défient l'ordre naturel. Nées dans les laboratoires maudits de l'Âge des Légendes ou surgies des cauchemars du Ténébreux, elles hantent les lieux abandonnés et traquent les vivants. Leur simple existence constitue une offense à la Création.

Ces entités ne suivent pas les lois biologiques ordinaires. Certaines se nourrissent de peur ou de douleur. D'autres existent partiellement hors de la réalité. Toutes portent en elles une part de la corruption qui ronge le monde depuis le Flétrissement.

\subsection{Créatures des Fléaux}

\textbf{Les \textit{Trollocs}} représentent la soldatesque de base des armées des Ténèbres. Ces hybrides grotesques mélangent traits humains et caractéristiques animales selon des combinaisons qui défient la nature. Leur chair putride exhale une puanteur reconnaissable à des lieues.

\emph{Stats~:} Physique +4, Mental -2, Social -4, Jauges~: 8

\emph{Capacités spéciales~:}
\begin{itemize}
\item \textbf{Vision nocturne}~: Aucun malus dans l'obscurité
\item \textbf{Résistance}~: Ignore les blessures légères (Marge 1-2)
\item \textbf{Peur animale}~: Les animaux fuient automatiquement
\end{itemize}

\emph{Faiblesses~:}
\begin{itemize}
\item \textbf{Symboles religieux}~: -2 à tous les tests en présence
\item \textbf{Lâcheté}~: Fuit si réduit à 3 Jauges ou moins
\end{itemize}

Leurs clans portent des noms d'animaux corrompus~: \textit{Dhai'mon}, \textit{Ahf'frait}, \textit{Ghar'ghael}. Chaque tribu développe des spécialisations martiales distinctes. Les \textit{Dhai'mon} excellent au corps à corps, les \textit{Kno'mon} manient des arcs primitifs mais efficaces.

\textbf{Les \textit{Myrddraal}} commandent les hordes trolloques avec une intelligence froide et calculatrice. Ces descendants dégénérés des humains ont perdu leurs yeux sans diminuer leur perception. Leur regard vide glace le sang des plus braves guerriers.

\emph{Stats~:} Physique +2, Mental +2, Social -2, Jauges~: 10

\emph{Capacités spéciales~:}
\begin{itemize}
\item \textbf{Perception paranormale}~: Détecte les êtres vivants sans les voir
\item \textbf{Déplacement d'ombre}~: Téléportation instantanée entre zones sombres
\item \textbf{Regard paralysant}~: Test \textbf{Dépassement + Esprit ≥ 14} ou paralysie 1 tour
\item \textbf{Lien \textit{Trolloc}}~: Commande télépathique, les \textit{Trollocs} meurent avec lui
\end{itemize}

\emph{Faiblesses~:}
\begin{itemize}
\item \textbf{Sensibilité au Pouvoir}~: Double dégâts des tissages offensifs
\item \textbf{Lames de Pouvoir}~: Vulnérabilité aux armes forgées magiquement
\end{itemize}

Leurs lames noires distillent un poison mortel qui résiste aux soins ordinaires. Seul le Pouvoir Unique peut neutraliser cette corruption métaphysique. Leur mort provoque la mort instantanée de tous les \textit{Trollocs} sous leur commandement direct.

\textbf{Les \textit{Draghkar}} chassent dans les airs avec une grâce mortelle. Ces prédateurs ailés utilisent leur chant hypnotique pour attirer leurs proies avant de drainer leur essence vitale. Leur beauté trompeuse dissimule une nature fondamentalement maléfique.

\emph{Stats~:} Physique +1, Mental +1, Social +3, Jauges~: 5

\emph{Capacités spéciales~:}
\begin{itemize}
\item \textbf{Vol}~: Déplacement aérien, +2 aux attaques depuis les airs
\item \textbf{Chant hypnotique}~: Test \textbf{Faconde + Esprit ≥ 12} ou approche volontaire
\item \textbf{Baiser de mort}~: Contact = 1 point de Jauge permanent perdu/tour
\item \textbf{Beauté surnaturelle}~: Déjà incluse dans Social +3
\end{itemize}

\emph{Faiblesses~:}
\begin{itemize}
\item \textbf{Attaques sonores}~: -4 à tous les tests sous cacophonie
\item \textbf{Fragilité au sol}~: -2 Combat si les ailes sont blessées
\end{itemize}

\subsection{Constructs de l'Âge des Légendes}

\textbf{Les \textit{Gholams}} incarnent la perfection artificielle au service de la destruction. Ces assassins créés par les généticiens de l'Âge des Légendes peuvent adopter n'importe quelle forme humaine. Leur chair synthétique résiste à toutes les agressions connues.

\emph{Caractéristiques Engrenages~:} Physique +3, Mental +2, Social +1

\emph{Capacités spéciales~:} Métamorphose, immunité au Pouvoir Unique, régénération, force surhumaine

\emph{Faiblesse~:} Vulnérabilité au \textit{foxhead medallion}, compulsions de création

Leur programmation originelle les lie à des objectifs spécifiques. Certains traquent les canalisateurs, d'autres protègent des sites secrets. Cette contrainte comportementale peut être exploitée par des adversaires astucieux.

\textbf{Les \textit{Jumara}} gardent les installations critiques depuis des millénaires. Ces créatures blindées combinent intelligence artificielle et prédation biologique. Leurs capteurs détectent les intrusions à travers plusieurs dimensions.

\emph{Caractéristiques Engrenages~:} Physique +4, Mental +1, Social -2

\emph{Capacités spéciales~:} Blindage énergétique, détection multidimensionnelle, armes intégrées, entretien automatique

\emph{Faiblesse~:} Protocoles de sécurité rigides, vulnérabilité aux impulsions électromagnétiques

\subsection{Entités Interdimensionnelles}

\textbf{Les Habitants du Vide} existent partiellement hors de la réalité normale. Ces prédateurs chassent à travers les dimensions, attirés par l'usage intensif du Pouvoir Unique. Leur forme véritable défie la perception humaine.

\emph{Caractéristiques Engrenages~:} Variable selon la manifestation

\emph{Capacités spéciales~:} Déplacement interdimensionnel, intangibilité partielle, drain énergétique, résistance magique

\emph{Faiblesse~:} Instabilité dimensionnelle, vulnérabilité aux \textit{ter'angreal} spécialisés

\textbf{Les Spectres du \textit{Blight}} hantent les territoires corrompus par les Ténèbres. Ces revenants gardent une conscience fragmentaire de leur existence passée. Leur présence refroidit l'air et fane la végétation.

\emph{Caractéristiques Engrenages~:} Physique +1, Mental -1, Social -3

\emph{Capacités spéciales~:} Intangibilité, drain de chaleur, terreur surnaturelle, résistance aux armes ordinaires

\emph{Faiblesse~:} Vulnérabilité à la lumière pure, attachement aux lieux de mort

\subsection{Créatures Corrompues}

\textbf{Les Vers Gris} infestent les cadavres laissés dans les zones de corruption. Ces parasites nécrophages transforment leurs hôtes en marionnettes putrides. Leur prolifération menace l'équilibre écologique des régions touchées.

\emph{Caractéristiques Engrenages~:} Physique +2, Mental -3, Social -4

\emph{Capacités spéciales~:} Animation des morts, prolifération parasitaire, résistance à la putréfaction, immunité à la douleur

\emph{Faiblesse~:} Vulnérabilité au feu, dépendance aux corps hôtes

\textbf{Les Chiens de l'Ombre} traquent leurs proies à travers les distances et les dimensions. Ces prédateurs surnaturels obéissent aux sorciers des Ténèbres les plus puissants. Leur loyauté absolue en fait des adversaires impitoyables.

\emph{Caractéristiques Engrenages~:} Physique +3, Mental +1, Social -2

\emph{Capacités spéciales~:} Pistage interdimensionnel, vitesse surnaturelle, morsure empoisonnée, pack instinct

\emph{Faiblesse~:} Dépendance au maître, vulnérabilité aux barrières spirituelles

\subsection{Survivants de l'Âge des Légendes}

\textbf{Les \textit{Aelfinn} et \textit{Eelfinn}} habitent leurs domaines respectifs au-delà des portes-pierres. Ces êtres anciens possèdent des connaissances vastes mais leur mentalité aliène rend tout échange périlleux. Leurs réponses cachent toujours des pièges subtils.

\emph{Caractéristiques Engrenages~:} Physique +1, Mental +4, Social +2

\emph{Capacités spéciales~:} Omniscience partielle, manipulation temporelle, création d'objets, résistance magique absolue

\emph{Faiblesse~:} Contraintes contractuelles, limitations dimensionnelles

\textbf{Les \textit{Nym}} gardent encore quelques jardins secrets de l'Âge des Légendes. Ces êtres végétaux intelligents incarnent l'harmonie entre technologie et nature. Leur extinction progressive symbolise la dégradation du monde.

\emph{Caractéristiques Engrenages~:} Physique +2, Mental +3, Social +1

\emph{Capacités spéciales~:} Contrôle végétal, guérison naturelle, symbiose écologique, longévité millénaire

\emph{Faiblesse~:} Vulnérabilité au Flétrissement, dépendance aux jardins-racines

\subsection{Conseils d'Utilisation pour le MJ}

Ces créatures ne sont jamais de simples obstacles mécaniques. Chacune porte une histoire, incarne un aspect de la corruption du monde, représente un défi spécifique pour les héros. Leur présence dans une aventure doit servir la narration autant que l'action.

Les \textit{Trollocs} attaquent en hordes désorganisées mais leur lâcheté peut être exploitée. Un \textit{Myrdraal} change radicalement la dynamique d'un combat, transformant la bestialité en tactique calculée. Son regard paralyse même les guerriers aguerris.

Les créatures de l'Âge des Légendes défient les attentes des personnages. Leurs capacités dépassent souvent la compréhension moderne. Leurs faiblesses reflètent les obsessions de leurs créateurs ou les contraintes de leur programmation.

L'usage du Pouvoir Unique contre ces entités produit des résultats variables. Certaines y résistent naturellement, d'autres s'en nourrissent. Cette imprévisibilité force les canalisateurs à développer des stratégies alternatives.

\subsection{Adaptation des Caractéristiques}

Le système Engrenages simplifie la gestion de ces créatures sans sacrifier leur spécificité. Leurs modificateurs reflètent leurs capacités principales tout en préservant l'équilibre du jeu. Les capacités spéciales remplacent les statistiques complexes par des effets narratifs clairs.

Les faiblesses ne sont jamais arbitraires. Elles découlent de la nature même de ces êtres, offrant aux joueurs ingénieux des moyens de surmonter des défis apparemment insurmontables. Cette cohérence interne renforce l'immersion et récompense la réflexion tactique.

\section{Galerie des Personnages Non-Joueurs}

\emph{``Dans ce monde en mutation, chaque visage cache une histoire, chaque rencontre peut changer le cours du Destin.''}

Le monde de La Roue du Temps grouille de personnages aux motivations diverses. Cette section présente des archétypes de PNJ organisés selon les \textbf{quatre forces majeures} qui influencent les événements~: le Pouvoir, le Peuple, le Commerce et les Ombres. Ces profils servent de base pour créer rapidement des rencontres mémorables et des adversaires crédibles.

\subsection{Le Pouvoir - Ceux qui Gouvernent}

Ces personnages représentent l'autorité établie~: nobles, militaires, Aes Sedai, magistrats et tous ceux qui détiennent un pouvoir officiel sur autrui.

\subsubsection{Dame Noble Andorienne}
\emph{Représentante de l'aristocratie traditionnelle}

\textbf{Concept}~: Politique courtoise cachant une volonté de fer

\textbf{Stats}~: Physique +1, Mental -3 (Argutie -6, Érudition -5), Social -2 (Faconde -4)

\textbf{Traits}~: Sang noble (3), Réseau d'influence (2), Éducation raffinée (2)

\textbf{Équipement}~: Garde-robe somptueuse, bijoux de famille, correspondance chiffrée

\emph{Utilisation}~: Source d'informations politiques, mécène pour missions délicates, obstacle bureaucratique ou alliée influente selon les circonstances.

\emph{Particularité}~: \emph{Ses conversations incluent toujours trois niveaux de sens~: ce qu'elle dit, ce qu'elle sous-entend, et ce qu'elle cache vraiment.}

\subsubsection{Capitaine de la Garde Royale}
\emph{Officier d'élite au service de la Couronne}

\textbf{Concept}~: Soldat professionnel loyal et compétent

\textbf{Stats}~: Physique -4 (Combat -7, Athlétisme -5), Mental +1, Social +1

\textbf{Traits}~: Formation militaire d'élite (3), Loyauté absolue (2), Sens tactique (2)

\textbf{Équipement}~: Armure de qualité, épée d'apparat fonctionnelle, cheval de guerre

\emph{Utilisation}~: Contact militaire fiable, source de renseignements sur les menaces, adversaire honorable en cas de conflit politique.

\emph{Particularité}~: \emph{Ne ment jamais, mais peut rester silencieux. Sa parole donnée vaut contrat signé.}

\subsubsection{Aes Sedai de l'Ajah Grise}
\emph{Diplomate et médiatrice au service de la paix}

\textbf{Concept}~: Négociatrice experte cachant ses véritables objectifs

\textbf{Stats}~: Physique +2, Mental -3 (Argutie -6), Social -3 (Faconde -6), Magie -2 (Esprit -5, Eau -4)

\textbf{Traits}~: Serments du Bâton Liant (3), Agelessness (2), Autorité des Aes Sedai (3)

\textbf{Équipement}~: Bâton de voyage, \textit{angreal} discret, correspondance diplomatique

\emph{Utilisation}~: Médiatrice dans les conflits, source d'informations sur les événements continentaux, mentor ou manipulatrice selon ses objectifs.

\emph{Particularité}~: \emph{Chaque phrase est techniquement vraie, mais l'ensemble peut induire en erreur. Maîtresse de l'art de ne pas mentir tout en trompant.}

\subsubsection{Seigneur-Commandant des Enfants}
\emph{Chef militaire fanatique de la Lumière}

\textbf{Concept}~: Croisé inflexible combattant l'Ombre sous toutes ses formes

\textbf{Stats}~: Physique -3 (Combat -6, Dépassement -5), Mental 0, Social -1 (Faconde -4)

\textbf{Traits}~: Foi inébranlable (3), Méfiance de la magie (3), Charisme militaire (2)

\textbf{Équipement}~: Armure blanche impeccable, épée bénie, étendard des Enfants

\emph{Utilisation}~: Allié contre les forces des Ténèbres, adversaire des canalisateurs, source de conflits moraux pour les héros.

\emph{Particularité}~: \emph{Absolument incorruptible dans sa foi, ce qui peut le rendre aussi dangereux pour les innocents que pour les coupables.}

\subsection{Le Peuple - La Voix des Simples Gens}

Ces personnages incarnent les citoyens ordinaires~: artisans, paysans, aubergistes, et tous ceux qui font tourner le monde par leur labeur quotidien.

\subsubsection{Aubergiste des Terres de l'Ouest}
\emph{Pilier de la communauté locale}

\textbf{Concept}~: Homme pragmatique connaissant tous les secrets du voisinage

\textbf{Stats}~: Physique +1, Mental 0, Social -2 (Citadin -5, Faconde -5)

\textbf{Traits}~: Réseau local (3), Discrétion professionnelle (2), Bon vivant (1)

\textbf{Équipement}~: Auberge bien achalandée, cave à vins, registre des voyageurs

\emph{Utilisation}~: Source d'informations locales, refuge pour se cacher, contact pour recruter de l'aide.

\emph{Particularité}~: \emph{Sait tout ce qui se passe dans sa région, mais ne révèle ses secrets qu'aux clients qui paient bien et se comportent honorablement.}

\subsubsection{Forgeron de Village}
\emph{Artisan respecté et confident de la communauté}

\textbf{Concept}~: Homme simple aux mains habiles et au cœur droit

\textbf{Stats}~: Physique -3 (Dépassement -5, Combat -4), Mental -3 (Créativité -6), Social +1

\textbf{Traits}~: Maître artisan (3), Force physique (2), Respect villageois (2)

\textbf{Équipement}~: Forge bien équipée, outils de maître, armes de sa fabrication

\emph{Utilisation}~: Réparateur d'équipement, source de sagesse populaire, recruteur de milice locale en cas de danger.

\emph{Particularité}~: \emph{Peut forger des objets exceptionnels quand l'urgence le pousse. Son travail reflète l'honneur de celui qui le commande.}

\subsubsection{Sagesse de Village}
\emph{Guérisseuse traditionnelle et autorité morale}

\textbf{Concept}~: Femme d'expérience gardienne des traditions ancestrales

\textbf{Stats}~: Physique +1, Mental -4 (Rural -6, Folklore -5), Social -1 (Faconde -4), Magie -2 (Eau -5)

\textbf{Traits}~: Savoir traditionnel (3), Respect villageois (3), Intuition (2)

\textbf{Équipement}~: Herbes médicinales, grimoire familial, bâton de fonction

\emph{Utilisation}~: Source de soins et de conseils, dépositaire des légendes locales, opposition aux changements jugés néfastes.

\emph{Particularité}~: \emph{Ses remèdes fonctionnent aussi bien que la médecine savante. Son opposition peut rallier tout un village contre les héros.}

\subsubsection{Barde Itinérant}
\emph{Gardien des histoires et messager entre les nations}

\textbf{Concept}~: Artiste voyageur colportant nouvelles et légendes

\textbf{Stats}~: Physique +1, Mental -2 (Folklore -5), Social -4 (Représentation -6, Faconde -5)

\textbf{Traits}~: Mémoire prodigieuse (3), Bienvenu partout (2), Contacts étendus (2)

\textbf{Équipement}~: Instruments de musique, recueils de chansons, lettres de recommandation

\emph{Utilisation}~: Source d'informations sur les événements distants, guide culturel, moyen de transmettre des messages.

\emph{Particularité}~: \emph{Ses chansons influencent l'opinion publique. Une ballade bien placée peut changer la réputation d'un héros en héros ou en vilain.}

\subsection{Le Commerce - Les Moteurs de l'Économie}

Ces personnages vivent du négoce et du transport des marchandises~: marchands, capitaines, banquiers, et tous ceux qui font circuler les richesses.

\subsubsection{Marchand de Caravane}
\emph{Négociant expérimenté sillonnant les routes commerciales}

\textbf{Concept}~: Homme d'affaires prudent mais opportuniste

\textbf{Stats}~: Physique +1, Mental -2 (Argutie -5), Social -4 (Faconde -6, Citadin -5)

\textbf{Traits}~: Réseau commercial (3), Œil pour les affaires (3), Connaissance des routes (2)

\textbf{Équipement}~: Caravane bien gardée, marchandises diverses, lettres de crédit

\emph{Utilisation}~: Transport vers destinations lointaines, source d'informations commerciales, accès aux marchés fermés.

\emph{Particularité}~: \emph{Connaît la valeur réelle de tout objet au premier coup d'œil. Ses contacts s'étendent sur plusieurs nations.}

\subsubsection{Capitaine de Navire Marchand}
\emph{Marin expérimenté maîtrisant les voies fluviales}

\textbf{Concept}~: Navigateur pragmatique connaissant tous les ports

\textbf{Stats}~: Physique -1 (Combat -4), Mental -4 (Mer -7), Social -1 (Faconde -4)

\textbf{Traits}~: Maître navigateur (3), Réseau portuaire (2), Courage maritime (2)

\textbf{Équipement}~: Navire rapide, équipage loyal, cartes détaillées

\emph{Utilisation}~: Transport fluvial rapide, contacts portuaires, informations sur le trafic maritime.

\emph{Particularité}~: \emph{Son navire peut atteindre des endroits inaccessibles par terre. Ses hommes lui sont dévoués jusqu'à la mort.}

\subsubsection{Banquier Cairhienien}
\emph{Financier discret gérant des fortunes considérables}

\textbf{Concept}~: Homme d'affaires cultivé jouant le Grand Jeu par intérêt

\textbf{Stats}~: Physique +2, Mental -4 (Argutie -7, Érudition -5), Social -2 (Faconde -4, Citadin -5)

\textbf{Traits}~: Maître du Grand Jeu (3), Réseau financier (3), Discrétion absolue (2)

\textbf{Équipement}~: Coffres-forts sécurisés, correspondance chiffrée, gardes du corps

\emph{Utilisation}~: Source de financement, blanchiment d'argent, informations sur les flux économiques.

\emph{Particularité}~: \emph{Garde les secrets de ses clients plus jalousement que ses propres. Un mot de lui peut ruiner ou enrichir une Maison noble.}

\subsubsection{Contrebandier des Frontières}
\emph{Trafiquant agile évitant taxes et douanes}

\textbf{Concept}~: Homme libre vivant dans l'ombre des lois

\textbf{Compétences principales}~: Maraude 6, Rural 5, Combat 4, Animaux 4

\textbf{Traits}~: Connaissance des passages secrets (3), Réseau illégal (2), Réflexes rapides (2)

\textbf{Équipement}~: Chevaux rapides, marchandises cachées, armes de qualité

\emph{Utilisation}~: Transport discret, accès aux objets interdits, contacts dans les milieux illégaux.

\emph{Particularité}~: \emph{Peut faire passer n'importe quoi, n'importe où, pour le bon prix. Sa liberté vaut plus que l'or.}

\subsection{Les Ombres - Servants des Secrets}

Ces personnages évoluent dans la clandestinité~: espions, assassins, Amis du Ténébreux, criminels et tous ceux qui prospèrent dans l'ombre.

\subsubsection{Espion de la Tour Blanche}
\emph{Agent secret au service des Aes Sedai}

\textbf{Concept}~: Observateur discret infiltré dans la société

\textbf{Compétences principales}~: Maraude 6, Argutie 5, Citadin 5, Faconde 4

\textbf{Traits}~: Couverture parfaite (3), Réseau d'informateurs (3), Formation secrète (2)

\textbf{Équipement}~: Identités multiples, moyens de communication sécurisés, armes cachées

\emph{Utilisation}~: Source d'informations sensibles, surveillance des héros, élimination discrète des obstacles.

\emph{Particularité}~: \emph{Sa véritable identité peut surprendre. Même ses proches ignorent sa vraie fonction.}

\subsubsection{Ami du Ténébreux Infiltré}
\emph{Serviteur secret des forces obscures}

\textbf{Concept}~: Homme ordinaire corrompu par l'appât du pouvoir

\textbf{Compétences principales}~: Variable selon la couverture, Maraude 4, Esprit 3 (pollué)

\textbf{Traits}~: Couverture sociale (3), Obéissance au Ténébreux (3), Corruption cachée (2)

\textbf{Équipement}~: Apparence normale, moyens de communication ténébreux, armes empoisonnées

\emph{Utilisation}~: Infiltration des organisations légales, sabotage des missions héroïques, révélation dramatique.

\emph{Particularité}~: \emph{Paraît parfaitement normal jusqu'à la révélation. Sa corruption peut contaminer son entourage.}

\subsubsection{Assassin Professionnel}
\emph{Tueur à gages maîtrisant l'art de la mort discrète}

\textbf{Concept}~: Artisan de la mort perfectionnant son art mortel

\textbf{Compétences principales}~: Combat 6, Maraude 7, Athlétisme 5, Dépassement 4

\textbf{Traits}~: Maître assassin (3), Insensibilité émotionnelle (2), Réseau criminel (2)

\textbf{Équipement}~: Armes spécialisées, poisons variés, déguisements multiples

\emph{Utilisation}~: Adversaire mortel, révélateur de complots, contact dans les bas-fonds.

\emph{Particularité}~: \emph{Ne tue jamais sans contrat. Son professionnalisme peut paradoxalement le rendre honorable à sa façon.}

\subsubsection{Chef de Guilde des Voleurs}
\emph{Dirigeant criminel organisant les activités illégales}

\textbf{Concept}~: Businessman du crime gérant un empire souterrain

\textbf{Compétences principales}~: Faconde 5, Argutie 6, Maraude 5, Citadin 6

\textbf{Traits}~: Autorité criminelle (3), Réseau étendu (3), Pragmatisme absolu (2)

\textbf{Équipement}~: Repaire sécurisé, sbires loyaux, informations compromettantes

\emph{Utilisation}~: Contrôleur du marché noir, source d'informations sur les activités illégales, obstacle ou allié selon les intérêts.

\emph{Particularité}~: \emph{Dirige son organisation comme une entreprise. Peut être plus fiable qu'un noble, ses accords étant garantis par la peur.}

\subsection{Conseils d'Utilisation}

Ces archétypes servent de base modulable selon les besoins de votre histoire. Adaptez leurs compétences, modifiez leurs traits, changez leurs motivations pour créer des personnages uniques tout en conservant leur essence fonctionnelle.

Chaque faction apporte sa dynamique propre aux intrigues. Le Pouvoir impose sa volonté, le Peuple exprime ses émotions, le Commerce cherche le profit, les Ombres manipulent en secret. L'interaction entre ces forces crée la richesse dramatique de l'univers.

\emph{``Dans ce monde complexe, même vos alliés cachent leurs véritables motifs. Seul celui qui comprend les quatre forces peut espérer naviguer dans les eaux troubles de La Roue du Temps.''}

\section{Les Réprouvés - Maîtres de l'Ombre}

\emph{``Treize d'entre les plus puissants ont prêté allégeance au Ténébreux dans l'Âge des Légendes. Emprisonnés avec leur maître, ils sont maintenant libres de répandre à nouveau leur corruption.''}

Les Réprouvés (\textit{Forsaken}) représentent l'élite absolue des servants du Ténébreux. Anciens Aes Sedai de l'Âge des Légendes qui ont trahi la Lumière, ils possèdent une maîtrise du Pouvoir Unique qui dépasse tout ce que le monde moderne peut concevoir. Leur retour annonce l'approche de la Dernière Bataille.

\subsection{Lanfear - Fille de la Nuit}
\emph{La plus puissante en \textit{saidar} parmi les \textit{Forsaken}}

\textbf{Concept}~: Ancienne amante de \textit{Lews Therin}, obsédée par la reconquête du Dragon

\textbf{Stats}~: Physique +1, Mental -8, Social -7, Magie -12, Jauges~: 20

\textbf{Capacités spéciales}~:
\begin{itemize}
\item \textbf{Maîtrise du Rêve}~: Contrôle total de \textit{Tel'aran'rhiod}, peut tuer dans les rêves
\item \textbf{\textit{Saidar} Suprême}~: Force 12 en canalisation, tissages impossibles pour les modernes
\item \textbf{Manipulation Émotionnelle}~: Compulsion subtile, détection des émotions
\item \textbf{Invocation}~: Peut créer des \textit{Draghkar} et autres servants de l'Ombre
\end{itemize}

\textbf{Faiblesses}~:
\begin{itemize}
\item \textbf{Obsession}~: -4 à tous les tests quand Rand/\textit{Lews Therin} est impliqué
\item \textbf{Orgueil}~: Sous-estime systématiquement ses adversaires
\item \textbf{Rivalité}~: Haine mortelle envers les autres \textit{Forsaken} femelles
\end{itemize}

\emph{Utilisation}~: Antagoniste majeur spécialisé dans la manipulation psychologique et les attaques oniriques. Ses motivations personnelles la rendent imprévisible même pour le Ténébreux.

\subsection{Ishamael - Traqueur des Mensonges}
\emph{Philosophe de l'inévitabilité du mal}

\textbf{Concept}~: Premier des \textit{Forsaken}, partiellement libéré cycliquement

\textbf{Stats}~: Physique +1, Mental -9, Social -6, Magie -11, Jauges~: 25

\textbf{Capacités spéciales}~:
\begin{itemize}
\item \textbf{\textit{Ba'alzamon}}~: Peut prendre l'apparence du Ténébreux lui-même
\item \textbf{Libération Cyclique}~: Influence le monde tous les mille ans environ
\item \textbf{Feu de l'Ombre}~: Flammes noires qui brûlent l'âme elle-même
\item \textbf{Maître du Destin}~: Peut manipuler les fils du Dessin avec le Pouvoir Véritable
\end{itemize}

\textbf{Faiblesses}~:
\begin{itemize}
\item \textbf{Folie Ancienne}~: Emprisonnement imparfait l'a rendu instable
\item \textbf{Nihilisme}~: Sa philosophie peut le pousser à des actes autodestructeurs
\end{itemize}

\emph{Utilisation}~: L'adversaire ultime avant la Dernière Bataille. Son influence s'étend sur des millénaires, manipulant les événements depuis l'ombre.

\subsection{Rahvin - Maître de la Compulsion}
\emph{Corrupteur des cours et destructeur des royaumes}

\textbf{Concept}~: Spécialiste de l'infiltration politique et de la manipulation mentale

\textbf{Stats}~: Physique +2, Mental -7, Social -8, Magie -9, Jauges~: 18

\textbf{Capacités spéciales}~:
\begin{itemize}
\item \textbf{Compulsion Parfaite}~: Contrôle mental indétectable par les victimes
\item \textbf{Charisme Surnaturel}~: +6 dans toutes les interactions sociales
\item \textbf{Infiltration Politique}~: Peut renverser un royaume de l'intérieur
\item \textbf{Foudre Destructrice}~: Tissages offensifs d'une puissance dévastatrice
\end{itemize}

\textbf{Faiblesses}~:
\begin{itemize}
\item \textbf{Impatience}~: Préfère les solutions rapides aux plans à long terme
\item \textbf{Dépendance}~: Addiction aux plaisirs charnels et au pouvoir politique
\end{itemize}

\emph{Utilisation}~: Parfait pour les intrigues de cour et la corruption des alliés. Transforme les campagnes politiques en cauchemars.

\subsection{Sammael - Le Destructeur d'Espoir}
\emph{Général de l'Ombre et rival éternel de \textit{Lews Therin}}

\textbf{Concept}~: Tacticien militaire consumé par la jalousie envers le Dragon

\textbf{Stats}~: Physique +3, Mental -8, Social -5, Magie -10, Jauges~: 22

\textbf{Capacités spéciales}~:
\begin{itemize}
\item \textbf{Génie Tactique}~: +8 dans tous les tests de stratégie militaire
\item \textbf{Commandement de l'Ombre}~: Contrôle des armées de \textit{Trollocs} et \textit{Myrddraal}
\item \textbf{Portails Instantanés}~: Voyage par les Voies et tissages de Voyage
\item \textbf{Destruction Massive}~: Spécialisé dans les tissages de guerre à grande échelle
\end{itemize}

\textbf{Faiblesses}~:
\begin{itemize}
\item \textbf{Cicatrices de Guerre}~: Blessures de l'Âge des Légendes limitent sa force physique
\item \textbf{Obsession Rivale}~: Fixation maladive sur la comparaison avec \textit{Lews Therin}
\end{itemize}

\emph{Utilisation}~: Chef de guerre parfait pour les campagnes militaires d'envergure. Sa rivalité avec Rand peut être exploitée.

\section{Héros de la Lumière}

\emph{``La Roue tisse comme elle l'entend, et parfois elle place les fils nécessaires exactement où ils doivent être.''}

Ces profils représentent les grands héros de l'époque actuelle à leur apogée. Ils servent de référence pour mesurer la puissance des PJ et peuvent intervenir comme alliés ou rivaux selon les circonstances.

\subsection{Rand al'Thor - Le Dragon Réincarné}
\emph{\textit{Ta'veren} Suprême et Champion de la Lumière}

\textbf{Concept}~: Berger devenu sauveur du monde, porteur du destin

\textbf{Stats}~: Physique -4, Mental -6, Social -3, Magie -9, Jauges~: 15

\textbf{Capacités spéciales}~:
\begin{itemize}
\item \textbf{\textit{Ta'veren} Majeur}~: Modifie la probabilité dans un rayon de plusieurs kilomètres
\item \textbf{Maîtrise d'Épée}~: +5 Combat avec les Formes d'épée
\item \textbf{\textit{Saidin} Puissant}~: Force 9 en canalisation masculine
\item \textbf{\textit{Lews Therin}}~: Accès aux souvenirs et compétences de la vie passée
\end{itemize}

\textbf{Faiblesses}~:
\begin{itemize}
\item \textbf{Folie Menaçante}~: Risque constant de sombrer dans la démence
\item \textbf{Fardeau du Destin}~: -2 dans les situations personnelles normales
\item \textbf{Cibles Multiples}~: Tous les servants de l'Ombre le traquent
\end{itemize}

\emph{Utilisation}~: PNJ de dernier recours ou rival indirect. Sa simple présence change la nature des événements.

\subsection{Moiraine Damodred - Aes Sedai de l'Ajah Bleue}
\emph{Tisseur de destins et guide des héros}

\textbf{Concept}~: Manipulatrice bienveillante dédiée à la victoire finale

\textbf{Stats}~: Physique +2, Mental -6, Social -5, Magie -7, Jauges~: 12

\textbf{Capacités spéciales}~:
\begin{itemize}
\item \textbf{Prescience Limitée}~: Visions partielles du futur à travers les \textit{ter'angreal}
\item \textbf{Réseau d'Espions}~: Contacts dans toutes les nations connues
\item \textbf{\textit{Saidar} Contrôlé}~: Efficacité maximale avec un pouvoir moyen
\item \textbf{Sacrifice Héroïque}~: Peut échanger sa vie contre un avantage décisif
\end{itemize}

\textbf{Faiblesses}~:
\begin{itemize}
\item \textbf{Serments Contraignants}~: Limitée par les Trois Serments des Aes Sedai
\item \textbf{Information Partielle}~: Ses plans reposent sur des visions incomplètes
\end{itemize}

\emph{Utilisation}~: Mentor parfait et source d'informations. Ses conseils cryptiques guident les héros.

\subsection{Lan Mandragoran - Épée de Malkier}
\emph{Garde-lié et dernier roi d'une nation morte}

\textbf{Concept}~: Guerrier parfait lié par l'honneur et le devoir

\textbf{Stats}~: Physique -6, Mental -3, Social -2, Magie 0, Jauges~: 14

\textbf{Capacités spéciales}~:
\begin{itemize}
\item \textbf{Lien de Garde}~: Partage de Force/endurance avec son Aes Sedai
\item \textbf{Maître d'Armes}~: +7 Combat avec toutes les armes blanches
\item \textbf{Sens du Danger}~: Détection automatique des embuscades
\item \textbf{Survie Extrême}~: Peut fonctionner sans sommeil pendant des semaines
\end{itemize}

\textbf{Faiblesses}~:
\begin{itemize}
\item \textbf{Pas de Magie}~: Aucune capacité de canalisation
\item \textbf{Devoir Contraignant}~: Doit protéger son Aes Sedai avant tout
\end{itemize}

\emph{Utilisation}~: Allié militaire idéal et entraîneur en combat. Son code d'honneur le rend prévisible.

\subsection{Conseils d'Utilisation pour les Légendes}

Ces personnages représentent le sommet de la puissance dans l'univers de La Roue du Temps. Leur utilisation demande des précautions particulières~:

\textbf{Pour les Réprouvés}~:
\begin{itemize}
\item Ne jamais les affronter directement sans préparation majeure
\item Leurs plans s'étendent sur des décennies ou des siècles
\item Chacun a sa spécialité et sa faiblesse exploitable
\item Leur arrogance de l'Âge des Légendes peut être retournée contre eux
\end{itemize}

\textbf{Pour les Héros}~:
\begin{itemize}
\item Ils interviennent rarement directement dans les affaires des PJ
\item Leur présence change automatiquement l'ampleur des enjeux
\item Leurs objectifs dépassent souvent les préoccupations immédiates
\item Ils peuvent être des rivaux autant que des alliés
\end{itemize}

\emph{``Face à de telles puissances, la ruse et la chance valent mieux que la force brute. Mais parfois, seul le courage peut faire la différence.''}
