\section{Glossaire et Références}

\subsection{Introduction au Glossaire}

Ce glossaire compile les termes essentiels de l'univers de La Roue du Temps selon leur traduction française officielle. Chaque entrée respecte la terminologie canonique tout en fournissant le contexte nécessaire à son usage en jeu de rôle. Les références croisées facilitent la navigation entre concepts liés.

Les termes en \textit{italique} suivent les conventions typographiques établies pour préserver la cohérence avec l'œuvre originale. Cette attention aux détails renforce l'immersion et respecte l'intention de l'auteur.

\medskip
\noindent\rule{\textwidth}{0.4pt}

\subsection{A}

\begin{description}[style=multiline,leftmargin=3cm]

\item[\textbf{Aes Sedai}] \textit{(invariable)} Femmes capables de canaliser le \textit{saidar}, organisées en sept Ajahs selon leurs spécialisations. Liées par les Trois Serments qui limitent l'usage de leurs pouvoirs. Basées à Tar Valon sous l'autorité de l'Amyrlin. Respectées et craintes à travers le monde pour leur longévité et leurs capacités.\\
\textit{Voir aussi~: Ajah, Amyrlin, Serments}

\item[\textbf{Âge des Légendes}] Période de paix et de prospérité technologique précédant le Flétrissement du Monde. Caractérisée par la maîtrise complète du Pouvoir Unique et l'absence de guerre. Société utopique détruite par la folie de Lews Therin Telamon et l'intervention du Ténébreux.\\
\textit{Voir aussi~: Flétrissement du Monde, Lews Therin}

\item[\textbf{Ajah}] Organisation interne des Aes Sedai selon leur philosophie~:
\begin{itemize}[topsep=0pt,itemsep=2pt]
  \item \textbf{Rouge}~: Traque les hommes capables de canaliser
  \item \textbf{Bleue}~: Défend les causes justes et combat l'injustice
  \item \textbf{Verte}~: Se prépare à la Dernière Bataille
  \item \textbf{Jaune}~: Se spécialise dans la Guérison
  \item \textbf{Blanche}~: Poursuit la logique et la philosophie pure
  \item \textbf{Brune}~: Préserve et recherche la connaissance
  \item \textbf{Grise}~: Pratique la médiation et la diplomatie
\end{itemize}

\item[\textbf{Amis des Ténèbres}] Humains ayant choisi de servir le Ténébreux en échange de pouvoir ou d'immortalité. Organisés en cellules secrètes infiltrant toutes les couches de la société. Pratiquent rituels blasphématoires et sacrifices humains. Identifiables par leur marque invisible aux yeux ordinaires.\\
\textit{Voir aussi~: Ténébreux, Réprouvés}

\item[\textbf{Amyrlin (Siège d')}] Dirigeante suprême des Aes Sedai et de la Tour Blanche. Élue par la Salle de la Tour, elle transcende théoriquement les divisions d'Ajah. Dispose de pouvoirs considérables mais reste soumise à la loi et aux traditions de la Tour.

\item[\textbf{\textit{Angreal}}] Objets de l'Âge des Légendes permettant à un canalisateur d'atteindre plus de pouvoir qu'il ne pourrait naturellement. Existent en versions féminines (\textit{saidar}) et masculines (\textit{saidin}). Relativement rares et précieux, souvent gardés par les Aes Sedai.\\
\textit{Voir aussi~: Sa'angreal, Ter'angreal}

\item[\textbf{Asha'man}] Hommes formés pour canaliser le \textit{saidin} sous l'autorité de Mazrim Taim. Basés à la Tour Noire près de Caemlyn. Organisés militairement avec grades de Soldat, Dévoué et Asha'man complet. Redoutés pour leur efficacité au combat mais suspectés de corruption.

\end{description}

\subsection{B}

\begin{description}[style=multiline,leftmargin=3cm]

\item[\textbf{Balefire}] \textit{(bûchefeu)} Weave destructeur qui efface rétroactivement l'existence de sa cible. Plus le flux est puissant, plus loin dans le temps l'effacement remonte. Usage interdit par toutes les organisations de canalisateurs car il endommage le Dessin lui-même.

\item[\textbf{Barrage}] Obstacle créé par le Pouvoir Unique pour bloquer l'accès au \textit{saidin} ou \textit{saidar} d'un canalisateur. Technique complexe exigeant plusieurs canalisateurs coordonnés. Effet temporaire mais peut être renouvelé indéfiniment.

\end{description}

\subsection{C}

\begin{description}[style=multiline,leftmargin=3cm]

\item[\textbf{Canaliser}] Acte de puiser dans le Pouvoir Unique (\textit{saidin} ou \textit{saidar}) pour créer des effets surnaturels. Capacité innée touchant environ 1\% de la population. Requiert apprentissage et pratique pour éviter les dangers mortels.

\item[\textbf{Coupé (du Pouvoir)}] État d'un canalisateur définitivement privé d'accès au Pouvoir Unique. Peut résulter d'un traumatisme, d'une punition délibérée, ou d'un accident. Souvent accompagné de dépression profonde et de tendances suicidaires.

\end{description}

\subsection{D}

\begin{description}[style=multiline,leftmargin=3cm]

\item[\textbf{\textit{Damane}}] Femmes capables de canaliser asservies par l'Empire Seanchan. Contrôlées par des \textit{sul'dam} via des \textit{a'dam}. Considérées comme des animaux domestiques dangereux nécessitant dressage constant. Conditionnées dès l'enfance à accepter leur servitude.

\item[\textbf{Dessin (Le)}] Trame complexe tissée par la Roue du Temps, déterminant les événements et destinées. Peut être influencé par les \textit{ta'veren} mais jamais complètement contrôlé. Les Prophéties révèlent certains motifs du Dessin sans en dévoiler tous les détails.

\end{description}

\subsection{E}

\begin{description}[style=multiline,leftmargin=3cm]

\item[\textbf{Enfants de la Lumière}] Organisation militaire et religieuse dédiée à combattre les Ténèbres sous toutes ses formes. Basés à Amador sous l'autorité du Seigneur Capitaine Commandeur. Reconnaissables à leurs capes blanches et leur fanatisme. Suspectent les Aes Sedai de servir les Ténèbres.\\
\textit{Voir aussi~: Questionneurs, Capes Blanches}

\end{description}

\subsection{F}

\begin{description}[style=multiline,leftmargin=3cm]

\item[\textbf{Flétrissement du Monde}] Catastrophe cosmique ayant mis fin à l'Âge des Légendes. Causée par l'attaque ratée contre la prison du Ténébreux, elle a brisé le monde géographiquement et socialement. A également contaminé le \textit{saidin}, rendant fous tous les hommes qui canalisent.

\item[\textbf{Flétrissure (La)}] Territoire désolé aux frontières nord du monde civilisé, créé par la corruption persistante du Ténébreux. Abrite les Trollocs, Myrdraal et autres créatures des Ténèbres. S'étend ou se rétracte selon la force relative de la Lumière et des Ténèbres.

\end{description}

\subsection{G}

\begin{description}[style=multiline,leftmargin=3cm]

\item[\textbf{Grande Serpent}] Symbole des Aes Sedai représentant l'éternité et les cycles du temps. Gravé sur l'anneau porté par toutes les initiées de la Tour Blanche. Évoque la Roue du Temps et l'infinité des cycles d'existence.

\item[\textbf{Gris (Hommes)}] Hommes ayant commencé à canaliser le \textit{saidin} souillé. La folie les gagne progressivement, accompagnée de pourriture physique. Généralement traqués et exécutés par les Aes Sedai Rouges avant qu'ils ne deviennent trop dangereux.

\end{description}

\subsection{L}

\begin{description}[style=multiline,leftmargin=3cm]

\item[\textbf{Lié (Lige)}] Homme lié magiquement à une Aes Sedai par le biais d'un weave spécial. Acquiert endurance, rapidité et résistance surhumaines en échange d'un dévouement absolu. Peut ressentir émotions et état de santé de son Aes Sedai.

\item[\textbf{Lumière (La)}] Force cosmique opposée aux Ténèbres, incarnant l'ordre, la vie et la création. Ne possède pas de personnalité consciente contrairement au Ténébreux. Se manifeste à travers les actions des individus vertueux et l'équilibre naturel du monde.

\end{description}

\subsection{M}

\begin{description}[style=multiline,leftmargin=3cm]

\item[\textbf{Mashadar}] Corruption spécifique à Shadar Logoth, née de la haine concentrée contre les Ténèbres. Dévore indifféremment serviteurs de la Lumière et des Ténèbres. Se manifeste sous forme de brouillard noir animé d'une volonté malveillante primitive.

\item[\textbf{Myrddraal}] \textit{(Fléau)} Créatures humanoïdes sans yeux servant d'officiers aux armées de Trollocs. Capables de se déplacer dans l'ombre et de paralyser par leur regard vide. Forgent des lames noires au poison mortel. Craints même par les autres serviteurs des Ténèbres.

\end{description}

\subsection{P}

\begin{description}[style=multiline,leftmargin=3cm]

\item[\textbf{Pouvoir Unique (Le)}] Force fondamentale animant la Création, divisée en moitiés masculine (\textit{saidin}) et féminine (\textit{saidar}). Source de tous les phénomènes surnaturels. Accessible seulement aux individus nés avec cette capacité innée.

\end{description}

\subsection{Q}

\begin{description}[style=multiline,leftmargin=3cm]

\item[\textbf{Questionneurs}] Bras inquisitorial des Enfants de la Lumière, spécialisé dans l'extraction d'aveux par la torture. Dirigés par le Grand Inquisiteur, ils opèrent avec autonomie considérable. Redoutés pour leurs méthodes impitoyables et leur fanatisme extrême.

\end{description}

\subsection{R}

\begin{description}[style=multiline,leftmargin=3cm]

\item[\textbf{Réprouvés (Les)}] Treize canalisateurs les plus puissants ayant rejoint le Ténébreux durant l'Âge des Légendes. Emprisonnés avec leur maître lors du Flétrissement du Monde. Possèdent connaissances et pouvoirs de l'ancienne époque. Immortels tant que le Ténébreux les protège.

\item[\textbf{Rêveurs}] Individus possédant la capacité naturelle d'entrer consciemment dans Tel'aran'rhiod. Talent extrêmement rare, principalement féminin. Peuvent influencer le Monde des Rêves par leur volonté mais risquent la mort en cas d'erreur.

\item[\textbf{Roue du Temps (La)}] Concept cosmologique représentant le cycle éternel de l'existence. Tisse le Dessin de toutes les vies à travers ses révolutions infinies. Les Âges se succèdent selon un ordre immuable, chaque événement possédant ses échos dans les Âges passés et futurs.

\end{description}

\subsection{S}

\begin{description}[style=multiline,leftmargin=3cm]

\item[\textbf{\textit{Sa'angreal}}] Versions les plus puissantes des \textit{angreal}, permettant d'atteindre des niveaux de Pouvoir dépassant largement les capacités humaines normales. Extrêmement rares et précieux. Leur usage inconsidéré peut détruire l'utilisateur ou déformer la réalité locale.

\item[\textbf{\textit{Saidar}}] Moitié féminine du Pouvoir Unique, accessible uniquement aux femmes. Se caractérise par la soumission et l'harmonie dans son maniement. Non souillée par la corruption du Ténébreux, contrairement au \textit{saidin}.

\item[\textbf{\textit{Saidin}}] Moitié masculine du Pouvoir Unique, souillée depuis le Flétrissement du Monde. Rend progressivement fous tous les hommes qui y puisent. Se caractérise par la lutte et la domination dans son contrôle.

\item[\textbf{Serments (Les Trois)}] Restrictions magiques liées par toutes les Aes Sedai lors de leur élévation~:
\begin{enumerate}[topsep=0pt,itemsep=2pt]
  \item Ne jamais proférer de mensonge
  \item Ne jamais forger d'arme pour tuer
  \item N'utiliser le Pouvoir comme arme que contre les Ténèbres, en défense de sa vie, ou de celle de son Lige
\end{enumerate}

\item[\textbf{Shadar Logoth}] Ruines de l'ancienne Aridhol, cité détruite par sa propre haine des Ténèbres. Hantée par Mashadar, corruption aussi dangereuse que celle du Ténébreux mais de nature opposée. Évitée par tous les êtres sensés.

\item[\textbf{\textit{Sul'dam}}] Femmes de l'Empire Seanchan dressées pour contrôler les \textit{damane} via les colliers \textit{a'dam}. Ignorent généralement qu'elles possèdent elles-mêmes la capacité latente de canaliser. Considérées comme nobles et respectées dans la hiérarchie seanchane.

\end{description}

\subsection{T}

\begin{description}[style=multiline,leftmargin=3cm]

\item[\textbf{\textit{Ta'veren}}] Individus autour desquels la Roue tisse le Dessin plus fortement. Leurs actions influencent les événements de manière disproportionnée. Attirent coïncidences, hasards improbables et retournements dramatiques. Phénomène rare, généralement lié aux périodes de grands changements.

\item[\textbf{\textit{Tel'aran'rhiod}}] \textit{(Monde des Rêves)} Dimension parallèle reflétant l'inconscient collectif de l'humanité. Accessible aux Rêveurs et via certains \textit{ter'angreal}. Les lois physiques y obéissent à la volonté plus qu'à la logique. La mort y est définitive et se répercute sur le corps physique.

\item[\textbf{Ténébreux (Le)}] Entité cosmique incarnant le chaos, la destruction et le mal absolu. Emprisonné depuis la Création mais ses influences corrompent le monde. Oppose au Créateur et à la Lumière. Cherche à détruire la Roue du Temps et l'existence elle-même.

\item[\textbf{\textit{Ter'angreal}}] Objets créés avec le Pouvoir Unique durant l'Âge des Légendes. Produisent effets variés sans nécessiter de canalisation active. Fonctionnement souvent mystérieux pour les utilisateurs modernes. Peuvent être utilisés par les non-canalisateurs selon leur conception.

\item[\textbf{Tour Blanche}] Forteresse et centre de formation des Aes Sedai située à Tar Valon. Symbole architectural de la puissance et de l'unité des Aes Sedai. Abrite la bibliothèque la plus complète du monde connu et de nombreux \textit{ter'angreal}.

\item[\textbf{Tour Noire}] Forteresse des Asha'man construite près de Caemlyn. Centre de formation militaire pour les canalisateurs masculins. Architecture austère reflétant la nature martiale de l'organisation.

\item[\textbf{Trollocs}] Créatures hybrides combinant traits humains et animaux, créées par Aginor durant la Guerre des Ténèbres. Forment la masse des armées des Ténèbres. Organisés en bandes selon leur héritage animal. Lâches individuellement mais redoutables en groupe.

\end{description}

\subsection{V}

\begin{description}[style=multiline,leftmargin=3cm]

\item[\textbf{Voie (La)}] Ensemble de tunnels et passages extra-dimensionnels créés durant l'Âge des Légendes. Permet voyage rapide entre points distants via des Portails-Pierres. Habitée par des créatures hostiles et soumise à des lois temporelles anarchiques.

\end{description}

\subsection{W}

\begin{description}[style=multiline,leftmargin=3cm]

\item[\textbf{Weave}] \textit{(Tissage)} Combinaison spécifique de flux du Pouvoir Unique créant un effet déterminé. Chaque weave possède sa propre signature énergétique et ses exigences techniques. L'apprentissage des tissages constitue l'essentiel de la formation des canalisateurs.

\end{description}

\medskip

Cette compilation respecte la terminologie française officielle tout en fournissant le contexte nécessaire à l'usage ludique. Les références croisées permettent d'approfondir la compréhension des concepts interconnectés de cet univers complexe.

\begin{citationwot}{``La connaissance précise des mots évite bien des malentendus... et des catastrophes.''}{Verin Mathwin}
\end{citationwot}
