\subsection{Principes du Combat}

Dans l'univers de La Roue du Temps, le combat transcende la simple violence pour devenir un art martial codifié. Les Formes d'épée transforment chaque duel en ballet mortel, tandis que les batailles rangées voient s'affronter tactiques millénaires et prodiges du Pouvoir Unique. Cette section adapte les règles d'\textit{Engrenages} aux spécificités de cet univers où l'honneur compte autant que l'efficacité.

\subsubsection{Initiative et Séquence}

L'initiative se détermine par un jet de $1\mathrm{d}6$ pour chaque protagoniste. Les modificateurs suivants peuvent s'appliquer :
\begin{itemize}
\item \textbf{Surprise} : Initiative automatique pour l'attaquant
\item \textbf{Terrain favorable} : +1 au dé d'initiative
\item \textbf{Arme plus longue} : +1 en première passe d'armes
\item \textbf{Formation militaire} : +2 pour les soldats entraînés
\end{itemize}

\textit{Exemple} : \textit{Gawyn charge l'épée haute depuis les escaliers du Palais Royal. Son \textbf{Combat} de 6 + \textbf{Athlétisme} de 4 + $2\mathrm{d}6$ bénéficient du modificateur de terrain (+2) et de surprise (+4). Son total de 16 + 6 = 22 lui assure d'agir en premier contre les gardes stupéfaits.}

\subsubsection{Actions par Tour}

Chaque tour de combat dure environ 3 secondes et permet :
\begin{itemize}
\item \textbf{Une action principale} : attaque, tissage complexe, déplacement important
\item \textbf{Actions mineures illimitées} : parade, esquive, déplacement tactique
\item \textbf{Actions réflexes} : interruption d'un tissage, contre-attaque
\end{itemize}

La fluidité du combat encourage l'enchaînement naturel des actions plutôt que le découpage mécanique.

\subsection{Résolution des Attaques}

\subsubsection{Attaque de Base}

Les attaques suivent la formule d'\textit{Engrenages} adaptée :
\textbf{Combat + Spécialisation + Traits + $2\mathrm{d}6$ + modificateurs $\geq$ 10}

Le combat suit les règles d'\textbf{opposition} d'\textit{Engrenages} :
\begin{itemize}
\item \textbf{Les deux adversaires} font leur jet simultannément
\item \textbf{PNJ} : valeur négative selon leur niveau (Physique -X)
\item \textbf{Protection d'armure} : modifie les dégâts reçus
\item \textbf{Bonus de Forme d'épée} : s'ajoute au jet d'attaque
\end{itemize}

\textit{Exemple} : \textit{Lan attaque un Myrddraal. Son \textbf{Combat} de 8 + \textbf{Spécialisation Épée} (+2) + \textbf{Trait Garde-Frontière} (+2) + $2\mathrm{d}6$ (résultat 7) = 19. Le Myrddraal (Physique -2) oppose sa défense. L'attaque réussit avec une bonne marge, infligeant des dégâts considérables.}

\subsubsection{Marges et Effets Spéciaux}

La \textbf{Marge de réussite} détermine l'impact :
\begin{itemize}
\item \textbf{Marge 0-2} : Touché simple, dégâts normaux
\item \textbf{Marge 3-5} : Touché net, +1 dégât, effet mineur
\item \textbf{Marge 6-8} : Touché excellent, +2 dégâts, effet notable
\item \textbf{Marge 9+} : Coup magistral, +3 dégâts, effet spectaculaire
\end{itemize}

\textbf{Effets spéciaux possibles} :
\begin{itemize}
\item \textbf{Désarmement} (Marge 6+) : l'arme de l'adversaire vole
\item \textbf{Renversement} (Marge 8+) : l'ennemi tombe au sol
\item \textbf{Intimider} (Marge 6+) : les témoins reculent, impressionnés
\item \textbf{Forme parfaite} (Marge 9+) : le mouvement devient légendaire
\end{itemize}

\textit{Exemple} : \textit{Avec sa Marge de 10, Lan exécute une technique d'épée parfaite. Son épée tranche la lame noire du Myrddraal en deux, et la perfection de son geste glace d'effroi les Trollocs témoins de la scène.}

\subsection{Armes et Équipements}

\subsubsection{Épées - L'Art Noble}

L'épée transcende le simple outil pour devenir symbole de statut et instrument d'art martial. Les \textbf{Formes d'épée} confèrent des bonus tactiques :

\textbf{Modificateurs d'armes} :
\begin{itemize}
\item \textbf{Épée courte} : +1 en espace confiné, -1 à cheval
\item \textbf{Épée longue} : polyvalente, aucun modificateur
\item \textbf{Épée à deux mains} : +2 dégâts, -2 en formation serrée
\item \textbf{Épée héron} : +1 attaque et défense (maîtres uniquement)
\end{itemize}

\textit{Exemple de technique} : Certaines manoeuvres avancées permettent d'attaquer plusieurs adversaires simultanément avec des malus répartis mais des bonus défensifs.

\subsubsection{Lances - Arme du Peuple}

\textbf{Lance courte} (Aiels) :
\begin{itemize}
\item Portée : corps à corps et proche
\item Bonus : +2 contre la cavalerie, +1 en formation
\item Malus : -3 en espace confiné
\end{itemize}

\textbf{Lance de cavalerie} :
\begin{itemize}
\item Bonus : +3 dégâts en charge, +2 portée
\item Malus : inutilisable au corps à corps
\end{itemize}

\textit{Exemple} : \textit{Un clan Aiel en formation de lances courtes bénéficie de +1 pour la coordination. Face à une charge de cavalerie cairhienaise, ils obtiennent +2 supplémentaires, rendant leur défense redoutable malgré leur équipement léger.}

\subsubsection{Arcs - Maîtres de la Distance}

\textbf{Portées et modificateurs} :
\begin{itemize}
\item \textbf{Courte} (0-100m) : +2 à l'attaque
\item \textbf{Moyenne} (100-300m) : aucun modificateur
\item \textbf{Longue} (300-500m) : -2 à l'attaque
\item \textbf{Extrême} (500m+) : -4 à l'attaque
\end{itemize}

\textbf{Spécialisations d'archer} :
\begin{itemize}
\item \textbf{Tir rapide} : deux flèches par action (-2 chacune)
\item \textbf{Tir de précision} : +3 attaque, une action de visée
\item \textbf{Tir en cloche} : ignore les couverts partiels
\end{itemize}

\textit{Exemple} : \textit{Birgitte tire à distance extrême sur un Myrddraal. Son \textbf{Combat} de 9 + \textbf{Spécialisation Arc} de 4 + \textbf{Trait Héros de la Corne} (+3) compensent largement le malus de -4. Sa flèche trouve sa cible avec une précision légendaire.}

\subsection{Techniques d'Épée - L'Art Martial}

\textit{Note : Les techniques d'épée décrites ici sont des adaptations pour le jeu de rôle, inspirées de l'univers mais non extraites directement des romans.}

L'art de l'épée constitue le cœur de la tradition martiale. Les combattants développent des techniques spécialisées offrant des avantages tactiques selon les situations.

\subsubsection{Techniques de Base}

\begin{center}
\begin{tabular}{|l|c|c|p{6cm}|}
\hline
\textbf{Technique} & \textbf{Coût XP} & \textbf{Prérequis} & \textbf{Effet} \\
\hline
\textbf{Défense Fluide} & 3 & Combat 3+ & +2 défense, esquives élégantes \\
\hline
\textbf{Garde Immuable} & 4 & Combat 4+ & +3 défense, -1 attaque, immunité feintes \\
\hline
\textbf{Frappe Multiple} & 5 & Combat 5+ & Attaque 2 adversaires, -2 chaque attaque \\
\hline
\textbf{Précision Mortelle} & 6 & Combat 6+ & Ignore armures lourdes (+2 vs protection 3+) \\
\hline
\end{tabular}
\end{center}

\subsubsection{Acquisition et Usage}

\textbf{Apprentissage :}
\begin{itemize}
\item \textbf{Maître requis} : Compétence Combat supérieure au personnage
\item \textbf{Coût} : Points d'expérience selon le tableau
\item \textbf{Temps} : 1 mois d'entraînement intensif par technique
\end{itemize}

\textbf{Utilisation en Combat :}
\begin{itemize}
\item \textbf{Choix tactique} : Déclaré au début de l'attaque
\item \textbf{Une technique par round} : Pas de cumul possible
\item \textbf{Fatigue} : Usage intensif (-1 après 5 rounds consécutifs)
\end{itemize}

\subsubsection{Niveaux de Maîtrise}

\begin{center}
\begin{tabular}{|c|l|p{7cm}|}
\hline
\textbf{Niveau Combat} & \textbf{Maîtrise} & \textbf{Capacités} \\
\hline
\textbf{1-3} & Apprenti & 1 technique, exécution -1 aux bonus \\
\hline
\textbf{4-6} & Pratiquant & 3 techniques, changement en combat \\
\hline
\textbf{7+} & Maître-Lame & Toutes techniques, création de variantes \\
\hline
\end{tabular}
\end{center}

\textbf{Trait spécial Maître-Lame :}
\begin{itemize}
\item \textbf{Épée-Héron} (5 XP) : +1 attaque et défense permanents
\end{itemize}

\textit{Exemple d'expertise} : \textit{Lan adapte sa technique selon la situation : mouvement fluide pour esquiver, garde solide pour bloquer, puis frappe précise pour terrasser l'adversaire. Sa maîtrise transforme le combat en art.}

\subsection{Combat et Pouvoir Unique}

L'intégration du Pouvoir Unique révolutionne l'art de la guerre. Un seul canalisateur expérimenté peut tenir tête à une centaine de soldats, mais reste vulnérable à un archer habile ou à un adversaire qui brise sa concentration.

\subsubsection{Tissages de Combat Courants}

\textbf{Attaques Élémentaires} :
\begin{itemize}
\item \textbf{Boule de Feu} (Feu 3) : $3\mathrm{d}6$ dégâts, portée 100m, zone d'effet 3m
\item \textbf{Foudre} (Air + Feu 4) : $4\mathrm{d}6$ dégâts, portée 200m, précise
\item \textbf{Lame d'Air} (Air 2) : Combat +3, invisible, coupe tout
\item \textbf{Projectiles de Terre} (Terre 2) : Combat +2, munitions illimitées
\end{itemize}

\textbf{Défenses Tissées} :
\begin{itemize}
\item \textbf{Barrière d'Air} (Air 3) : +4 Protection contre projectiles
\item \textbf{Blindage de Terre} (Terre 4) : +6 Protection, immunité feu normal
\item \textbf{Bouclier Élémentaire} (Tous 5) : immunité magique pendant 1 tour
\end{itemize}

\textit{Exemple de combat magique} : \textit{Moiraine dresse une Barrière d'Air (+4 Protection) contre les flèches trolloques, puis riposte par une Boule de Feu (Feu 6 + $2\mathrm{d}6$ = 14 vs Difficulté 10). Sa Marge de 4 inflige $3\mathrm{d}6$+1 dégâts dans un rayon de 3 mètres, carbonisant une douzaine d'assaillants.}

\subsubsection{Vulnérabilités des Canalisateurs}

\textbf{Interruption de Tissage} :
\begin{itemize}
\item \textbf{Douleur physique} : test de Dépassement + Esprit vs dégâts subis
\item \textbf{Choc émotionnel} : perte automatique du \textit{saidar} chez les femmes
\item \textbf{Musique discordante} : -2 à tous les tissages
\item \textbf{Stedding} : impossible de saisir Le Pouvoir
\end{itemize}

\textbf{Épuisement} :
\begin{itemize}
\item Chaque tissage coûte 1 point de Fatigue
\item À 0 Fatigue : malus cumulatif de -1 par tissage supplémentaire
\item Épuisement total : perte de conscience, risque de \textit{burnout}
\end{itemize}

\textit{Exemple de vulnérabilité} : \textit{Egwene tisse simultanément trois défenses, accumulant 3 points de Fatigue. Quand une flèche aielle l'effleure (2 dégâts), elle doit réussir un test de \textbf{Dépassement} (4) + \textbf{Esprit} (5) + $2\mathrm{d}6$ $\geq$ 12 pour maintenir ses tissages. L'échec brise sa concentration et laisse ses compagnes sans protection.}

\subsubsection{Tactiques Anti-Magie}

\textbf{Archers d'élite} :
\begin{itemize}
\item Ciblent prioritairement les canalisateurs
\item Tir groupé pour saturer les défenses
\item Flèches à pointe \textit{ter'angreal} (ignore les barrières)
\end{itemize}

\textbf{Combat rapproché} :
\begin{itemize}
\item Les tissages complexes deviennent impossibles
\item Seuls les reflexes de combat subsistent
\item L'épée redevient l'arbitre du destin
\end{itemize}

\subsection{Tactiques Militaires}

La guerre dans l'univers de La Roue du Temps mélange traditions séculaires et innovations liées au Pouvoir Unique. Chaque nation a développé ses propres doctrines, adaptées à sa géographie et ses ressources.

\subsubsection{Formations Nationales}

\textbf{Phalange Tarabonnaise} :
\begin{itemize}
\item Bonus : +3 en défense contre cavalerie
\item Formation : lances longues en rangs serrés
\item Faiblesse : vulnérable aux archers, inflexible
\end{itemize}

\textbf{Charge Saldaeenne} :
\begin{itemize}
\item Bonus : +4 dégâts en charge, +2 moral
\item Spécial : peut traverser les lignes ennemies
\item Risque : désorganisation si la charge échoue
\end{itemize}

\textbf{Archerie Deux-Rivières} :
\begin{itemize}
\item Bonus : +2 précision, portée accrue
\item Cadence : deux flèches par action sans malus
\item Tradition : formation dès l'enfance
\end{itemize}

\textit{Exemple tactique} : \textit{L'armée d'Andor déploie sa cavalerie en écran, masquant le mouvement de ses archers des Deux-Rivières. Quand les Shaidos chargent, la cavalerie s'ouvre et révèle 200 arcs tendus. La volée de flèches décime la première ligne aielle avant qu'elle ne puisse engager.}

\subsubsection{Commandement et Moral}

\textbf{Tests de Commandement} :
\begin{itemize}
\item \textbf{Faconde + Trait de leadership + $2\mathrm{d}6$ $\geq$ Difficulté}
\item Difficulté basée sur les circonstances :
\begin{itemize}
\item Victoire assurée : 8
\item Combat équilibré : 10
\item Situation désespérée : 14
\end{itemize}
\end{itemize}

\textbf{Effets du commandement réussi} :
\begin{itemize}
\item \textbf{Marge 0-2} : maintien du moral
\item \textbf{Marge 3-5} : +1 à tous les tests de combat
\item \textbf{Marge 6+} : +2 combat, +1 action supplémentaire
\end{itemize}

\textit{Exemple de leadership} : \textit{Bashere observe ses cavaliers hésiter face aux \textit{gholams}. Son test de \textbf{Faconde} (7) + \textbf{Grand Capitaine} (+3) + $2\mathrm{d}6$ (9) = 19 contre une Difficulté de 14. Sa Marge de 5 galvanise ses hommes : ``Pour la Reine-Dragon !'' Le cri de guerre se propage, donnant +1 combat à toute la cavalerie.}

\subsubsection{Guerre de Siège}

\textbf{Défenses Architecturales} :
\begin{itemize}
\item \textbf{Murs traditionnels} : +6 Protection, ralentit les assauts
\item \textbf{Tours de l'Âge des Légendes} : +10 Protection, effets \textit{ter'angreal}
\item \textbf{Fortifications Ogier} : indestructibles, conçues pour l'éternité
\end{itemize}

\textbf{Machines de Guerre} :
\begin{itemize}
\item \textbf{Catapultes} : $6\mathrm{d}6$ dégâts zone, portée 300m
\item \textbf{Mangonneaux} : $4\mathrm{d}6$ dégâts précis, portée 200m
\item \textbf{Tours de siège} : permet l'escalade, +4 aux assauts
\end{itemize}

\textbf{Innovation : Artillerie Magique} :
\begin{itemize}
\item \textbf{Canalisateurs coordonnés} : peuvent démolir n'importe quelle fortification
\item \textbf{Contre-magie} : les défenseurs ripostent par leurs propres canalisateurs
\item \textbf{Wards et protections} : \textit{ter'angreal} de défense des anciennes citadelles
\end{itemize}

\textit{Exemple de siège} : \textit{Le siège de Caemlyn voit s'affronter l'artillerie traditionnelle amadicienne et les canalisateurs andorans. Quand trois \textit{Aes Sedai} coordonnent leurs Tissages de Terre pour saper les murs, les assiégeants ripostent en concentrant leurs catapultes sur la Tour d'où opèrent les sœurs, forçant celles-ci à diviser leur attention entre attaque et défense.}

\subsection{Adversaires de l'Ombre}

Les servants du Ténébreux défient les lois naturelles par leur existence même. Combattre ces créatures exige courage, tactique et souvent un peu de chance.

\subsubsection{Trollocs - Infanterie Corrompue}

\textbf{Profil de combat} :
\begin{itemize}
\item \textbf{Combat} : 5, \textbf{Athlétisme} : 6, \textbf{Dépassement} : 4
\item \textbf{Protection} : 2 (cuir clouté), \textbf{Points de Vie} : 4
\item \textbf{Traits} : \textit{Force brutale} (+2 dégâts), \textit{Peur de l'eau}, \textit{Hiérarchie de meute}
\end{itemize}

\textbf{Tactiques trolloques} :
\begin{itemize}
\item Attaque en hordes désorganisées
\item +2 combat quand ils sont 3+ contre 1
\item Fuient si leur Myrddraal est tué
\item Cannibales : se nourrissent de leurs morts
\end{itemize}

\textit{Exemple de rencontre} : \textit{Six Trollocs aux museaux de sanglier chargent la patrouille shienare. Leur \textbf{Combat} de 5 + \textbf{Force brutale} (+2) + \textbf{Supériorité numérique} (+2) = 9 + $2\mathrm{d}6$. Seule la discipline militaire des Gardes-Frontière peut égaler cette sauvagerie.}

\subsubsection{Myrddraal - Officiers des Ténèbres}

\textbf{Profil de combat} :
\begin{itemize}
\item \textbf{Combat} : 8, \textbf{Tous Pouvoirs} : 3, \textbf{Maraude} : 7
\item \textbf{Protection} : 3 (armure noire), \textbf{Points de Vie} : 6
\item \textbf{Capacités spéciales} :
\begin{itemize}
\item \textbf{Regard Paralysant} : test d'\textbf{Esprit} + \textbf{Dépassement} $\geq$ 12 ou paralysie 1 tour
\item \textbf{Lame Thakan'dar} : mort automatique si blessure grave
\item \textbf{Voyage par l'Ombre} : téléportation instantanée zones sombres
\item \textbf{Lien Trolloc} : commande telepathique, partage blessures
\end{itemize}
\end{itemize}

\textit{Exemple de terreur} : \textit{Le Myrddraal fixe Perrin de ses orbites vides. Test de \textbf{Dépassement} (5) + \textbf{Esprit} (4) + $2\mathrm{d}6$ = 15 vs 12. Perrin résiste à la paralysie grâce à ses yeux dorés de Frère-Loup, mais ses compagnons humains se figent de terreur, incapables de lever leurs armes.}

\subsubsection{Draghkar - Prédateurs Aériens}

\textbf{Profil de combat} :
\begin{itemize}
\item \textbf{Vol} : 8, \textbf{Combat} : 4, \textbf{Représentation} : 9
\item \textbf{Protection} : 1 (peau épaisse), \textbf{Points de Vie} : 3
\item \textbf{Attaques spéciales} :
\begin{itemize}
\item \textbf{Chant Hypnotique} : test de \textbf{Faconde} + \textbf{Esprit} $\geq$ 14 ou approche volontaire
\item \textbf{Baiser de Mort} : aspire l'âme (1 Point de Vie permanent par tour)
\item \textbf{Vol Silencieux} : +4 en discrétion, attaque surprise
\end{itemize}
\end{itemize}

\textit{Exemple de chasse} : \textit{Le Draghkar survole le campement en chantant. Nynaeve échoue à son test de résistance (\textbf{Faconde} 3 + \textbf{Esprit} 6 + $2\mathrm{d}6$ = 12 vs 14) et s'avance vers la créature. Seule l'intervention de Lan, immunisé par son lien d'\textit{Aes Sedai}, peut la sauver du baiser mortel.}

\subsubsection{Tactiques de Groupe Contre l'Ombre}

\textbf{Formation Anti-Trolloc} :
\begin{itemize}
\item Archers en arrière, lanciers devant
\item Éliminer d'abord le Myrddraal (les Trollocs fuient)
\item Feu et lumière : les créatures de l'Ombre détestent les flammes
\end{itemize}

\textbf{Défense Anti-Draghkar} :
\begin{itemize}
\item Guetteurs avec bouchons d'oreille
\item Canalisateurs pour détection magique
\item Combat rapproché : les Draghkar sont fragiles au sol
\end{itemize}

\subsection{Blessures et Guérison}

Dans un monde où la violence est omniprésente, savoir gérer les blessures peut faire la différence entre la vie et la mort. La guérison magique, bien que miraculeuse, n'est pas sans prix.

\subsubsection{Système de Blessures}

\textbf{Niveaux de dégâts} :
\begin{itemize}
\item \textbf{Éraflure} (1 dégât) : -1 à toutes les actions
\item \textbf{Blessure Légère} (2 dégâts) : -2, saignement mineur
\item \textbf{Blessure Grave} (3 dégâts) : -4, risque d'infection
\item \textbf{Blessure Critique} (4+ dégâts) : inconscience, test de survie
\end{itemize}

\textbf{Points de Vie par Constitution} :
\begin{itemize}
\item Constitution 1-2 : 2 PV
\item Constitution 3-4 : 3 PV
\item Constitution 5-6 : 4 PV
\item Constitution 7+ : 5 PV
\end{itemize}

\textit{Exemple de blessure} : \textit{Mat reçoit un coup d'épée qui inflige 3 dégâts. Avec ses 4 Points de Vie, il tombe à 1 PV restant - blessure grave. Le malus de -4 affecte tous ses tests, et il risque l'infection sans soins.}

\subsubsection{Guérison Naturelle}

\textbf{Récupération par repos} :
\begin{itemize}
\item \textbf{8 heures de repos} : récupère 1 éraflure
\item \textbf{1 jour complet} : récupère 1 blessure légère
\item \textbf{1 semaine de repos} : récupère 1 blessure grave
\item \textbf{1 mois d'alitement} : récupère 1 blessure critique
\end{itemize}

\textbf{Soins mundains} :
\begin{itemize}
\item \textbf{Premiers secours} : test de \textbf{Rural} ou \textbf{Citadin} + \textbf{Traits médicaux}
\item \textbf{Succès} : arrête les saignements, +1 récupération
\item \textbf{Herbes médicinales} : bonus de +2 aux tests de soin
\item \textbf{Chirurgie} : peut traiter blessures graves (+4 Difficulté)
\end{itemize}

\textit{Exemple de soin} : \textit{Nynaeve soigne la blessure de Mat avec \textbf{Rural} (8) + \textbf{Sagesse des Deux-Rivières} (+3) + $2\mathrm{d}6$ = 17. Cette réussite excellente (Marge 7) arrête l'hémorragie et réduit le temps de guérison de moitié.}

\subsubsection{Guérison par le Pouvoir}

\textbf{Tissages de Guérison} :
\begin{itemize}
\item \textbf{Guérison Mineure} (Eau 2) : soigne 1 blessure légère, coût 1 Fatigue
\item \textbf{Guérison Majeure} (Eau 4) : soigne 1 blessure grave, coût 3 Fatigue
\item \textbf{Guérison Totale} (Eau 6 + Esprit 3) : guérit tout, coût 6 Fatigue
\item \textbf{Résurrection} (Tous 8) : ramène un mort récent, coût toute la Fatigue
\end{itemize}

\textbf{Limitations importantes} :
\begin{itemize}
\item \textbf{Épuisement du guérisseur} : risque d'effondrement
\item \textbf{Choc de guérison} : le patient peut s'évanouir
\item \textbf{Blessures à l'âme} : résistent à la magie (Shadar Logoth, lame Thakan'dar)
\item \textbf{Guérison forcée} : possible mais dangereuse pour les deux parties
\end{itemize}

\textit{Exemple de guérison magique} : \textit{Moiraine tisse la Guérison Majeure pour sauver Lan. Son \textbf{Eau} de 7 + $2\mathrm{d}6$ = 15 vs Difficulté 12 réussit, guérissant instantanément la blessure grave du Garde. Mais l'effort lui coûte 3 points de Fatigue, la laissant chancelante. La magie a un prix, même pour secourir un ami.}

\subsubsection{Blessures Spéciales}

\textbf{Lame Thakan'dar} :
\begin{itemize}
\item Empoisonnement progressif incurable
\item Seule la guérison magique peut ralentir les effets
\item Mort inévitable sans intervention divine
\end{itemize}

\textbf{Souillure de Shadar Logoth} :
\begin{itemize}
\item Corrompt les tissages de guérison
\item S'étend si traitée magiquement
\item Seule la guérison naturelle fonctionne
\end{itemize}

\textbf{Burn-out magique} :
\begin{itemize}
\item Perte définitive de la capacité de canaliser
\item Aucune guérison possible
\item Traumatisme psychologique permanent
\end{itemize}
