% Chapitre 5 : Le Pouvoir de l'Unique
% Transcrit depuis first-draft/chapitre05_pouvoir_unique.md

\section{MAGIE --- LE POUVOIR DE L'UNIQUE}

À travers les âges, \textit{Aes Sedai} et autres détenteurs du Pouvoir ont puisé dans cette source ancienne, utilisant ses énergies pour façonner le monde. Le Pouvoir de l'Unique est puisé dans la Vraie Source, qui est divisée en deux moitiés : \textit{saidin}, la moitié masculine, et \textit{saidar}, la moitié féminine. Les hommes ne peuvent accéder qu'à \textit{saidin}, tandis que les femmes ne peuvent accéder qu'à \textit{saidar}. La \textit{saidar} doit être embrassée avec douceur et soumission, comme pour se laisser submerger par un courant puissant tout en le dirigeant subtilement. Le \textit{saidin}, en revanche, doit être dominé et maîtrisé par la force de volonté, comme pour dompter une tempête sauvage. Autrefois, les canalisateurs utilisaient en coopération les deux moitiés pour bâtir de grandes choses. Mais juste avant son emprisonnement, Le Ténébreux souilla le \textit{saidin}, condamnant les hommes qui le canalisent à la folie, la disgrâce et la mort.

Pour tisser un sort avec les éléments du Pouvoir de l'Unique dans l'univers de ``La Roue du Temps'' en utilisant les règles d'\textsc{Engrenages}, vous pouvez suivre les étapes suivantes. Ces règles permettent de gérer la complexité et les effets des tissages tout en intégrant les mécanismes de base d'\textsc{Engrenages}.

\subsection{Utiliser les compétences élémentaires}

La Vraie Source est composée de cinq énergies élémentaires : Air, Terre, Feu, Eau et Esprit. Les canalisateurs puisent dans ces énergies, parfois en les combinant, pour créer des effets spécifiques. Par exemple, Air et Feu peuvent créer une boule de feu, tandis que Terre et Eau sont essentiels pour la guérison. Ces créations magiques naissent des tissages nés de l'entremêlement des fils d'énergie élémentaire.

Chaque canalisateur possède des affinités naturelles avec certains de ces éléments. Une \textit{Aes Sedai} peut exceller dans la manipulation de l'Air et du Feu, mais être moins douée avec l'Eau et la Terre. Les compétences reflètent ces affinités, tandis que les spécialisations vont marquer les domaines d'excellence de l'\textit{Aes Sedai}.

Pour réaliser un effet magique, le joueur décrit l'effet qu'il souhaite accomplir. Le Maître de Jeu lui indique quel(s) élément(s) sont à l'œuvre dans son tissage. Le joueur fait ensuite un jet de compétence dans lequel il va cumuler les valeurs des compétences qu'il possède pour chaque élément.

\begin{exemplebox}
Ellyandre souhaite guérir son lige qui s'est fait éventrer par un Trolloc ; elle possède +6 en Eau, +4 en Esprit et +2 en Terre. Le Maître de jeu lui dit que la guérison nécessite des fils d'Eau et de Terre. Elle va donc faire un jet de compétence à +8 (+6 en Eau et +2 en Terre).
\end{exemplebox}

\subsection{Le ciblage dans La Roue du Temps}

Contrairement aux règles de magie d'\textsc{Engrenages}, un utilisateur du Pouvoir de l'Unique ne peut pas se cibler lui-même. Par contre, il peut aider un autre canalisateur à faire un effet le concernant.

\begin{exemplebox}
Par exemple, dans la série TV, Alanna est ciblée de flèches, mais elle ne peut pas se soigner. En revanche, elle apprend aux filles Cauthon comment guérir et mêle son tissage au leur pour le guider et l'améliorer.
\end{exemplebox}

\subsection{Nouveaux modificateurs}

\subsubsection{Nombre d'éléments}

Tisser des fils d'énergie élémentaire est relativement simple lorsqu'ils ne mettent en jeu qu'un seul Élément. En revanche, plus on combine d'élément, plus le tissage est complexe, jusqu'à avoir besoin des 5 éléments comme le fameux tissage interdit du Malefeu, qui élimine les personnes touchées de La Roue. Voici les modificateurs à appliquer :

\begin{itemize}
\item \textbf{-4} : 3 éléments ou plus dans le tissage
\item \textbf{-2} : 2 éléments dans le tissage
\item \textbf{0} : 1 élément dans le tissage
\end{itemize}

\subsubsection{Connaissance du tissage}

Un tissage fonctionne comme un sortilège : la théorie permet de comprendre quels fils élémentaires mélanger et comment, mais la pratique fréquente est le seul critère pour garantir des effets magiques stables. Lorsqu'une Canalisatrice veut puiser dans le pouvoir de l'Unique, le Maître de Jeu détermine ce qu'elle connaît du tissage nécessaire pour réaliser l'effet souhaité.

\begin{itemize}
\item \textbf{-4 tissage sauvage} : rentrent dans cette catégorie les usages de magie instantanés, les utilisateurs du pouvoir qui n'ont pas conscience de leurs capacités, ou ceux qui ne veulent suivre aucune règle.
\item \textbf{-2 tissage inconnu} : le canalisateur souhaite créer un effet qu'il n'a jamais tenté auparavant, et qui n'a jamais été vu par d'autre.
\item \textbf{0 tissage étudié} : le canalisateur souhaite créer un effet qu'il n'a pas encore lui même pratiqué, mais qui lui a été enseigné par ses maîtres ou qu'il a pu voir faire par d'autres.
\item \textbf{+2 tissage pratiqué} : le canalisateur souhaite utiliser un tissage qu'il a déjà utilisé plusieurs fois en entrainement.
\item \textbf{+4 tissage maîtrisé} : le canalisateur souhaite utiliser un tissage qu'il a déjà expérimenté plusieurs fois en situation réelle (expérience).
\end{itemize}

De nombreux tissages puissants de l'Âge des Légendes ont été perdus lors de la Rupture du Monde. Ces ``tissages perdus'' ne survivent que dans des fragments de mémoire ou des textes obscurs. Certains ne sont connus que par les Rénégats ou le Dragon Réincarné lui-même.

Ces tissages perdus sont souvent entourés de mystère et de légendes. Des tissages comme la ``Flamme de Tar Valon'' ou la ``Tempête de Terre'' sont mentionnés dans d'anciens textes, mais peu savent encore comment les réaliser. Les canalisateurs les plus dévoués peuvent passer leur vie à redécouvrir ces connaissances anciennes.

\subsubsection{Collaborer sur un tissage}

Le lien est une méthode par laquelle les canalisateurs peuvent unir leurs forces, combinant leur pouvoir pour former des tissages plus puissants qu'ils ne pourraient réaliser individuellement. Une fois liés en cercle, les canalisateurs partagent leur pouvoir sous la direction d'un ``meneur'' qui contrôle le tissage.

Chaque canalisateur ajoute +2 au jet du meneur, seul le meneur fait le jet.

\subsection{Lumière et Ténèbres}

Le typage des dés influe plus intensément sur les utilisateurs du pouvoir de l'Unique.

\begin{itemize}
\item \textit{saidin} : pour les utilisateurs masculins qui accèdent à la source souillée, si le dé de Ténèbres est le plus haut, il s'ajoute à la marge de réussite ET au contrecoup. La marge de réussite est limitée par la compétence la plus basse du tissage utilisée, le contrecoup n'est bien sûr pas limité.
\item \textit{saidar} : pour les utilisateurs féminins, si le dé de Lumière est le plus haut, le dé s'ajoute à la marge de réussie OU diminue le contrecoup, en échange d'un point de Destinée.
\item \textit{vraie source} : les utilisateurs de la vraie source au troisième âge se limitent aux Rejetés, ce qui fait d'eux des adversaires extrêmement forts. Pour eux, le dé le plus haut (Lumière ou Ténèbres) s'ajoute à la marge de réussite ou diminue le contrecoup, tout le temps.
\end{itemize}

\subsection{Contrecoups}

Embrasser la Vraie Source procure une sensation d'euphorie incomparable, décrite par certains comme plus enivrante que n'importe quelle drogue. Cette sensation est si puissante qu'elle peut devenir addictive. Les canalisateurs qui maintiennent l'étreinte trop longtemps ou trop souvent risquent de développer une dépendance qui peut mener à des problèmes physiques et mentaux.

Pour les hommes, le danger est encore plus grand. La souillure sur le \textit{saidin} corrompt leur esprit et leur corps à chaque utilisation du Pouvoir, les menant inexorablement vers la folie et une mort douloureuse.

Si le canalisateur est un homme, voici des idées de contrecoups en rapport avec la souillure.

\begin{tenebresbox}{Manifestations mentales}
\begin{itemize}
\item Paranoïa croissante et méfiance irrationnelle envers les proches
\item Hallucinations visuelles et auditives, souvent des voix chuchotant des instructions sinistres
\item Pertes de mémoire, d'abord mineures puis s'étendant à des pans entiers de souvenirs
\item Accès de rage incontrôlable et violence spontanée
\item Développement d'identités secondaires ou multiples qui peuvent prendre le contrôle
\item Obsessions morbides pour des objets ou des personnes spécifiques
\item Incapacité à distinguer les rêves de la réalité
\item Perte progressive du raisonnement logique et des inhibitions sociales
\item Confusion croissante suivie d'une démence complète dans les stades avancés
\end{itemize}
\end{tenebresbox}

\begin{tenebresbox}{Manifestations physiques}
\begin{itemize}
\item Tremblements des mains puis des membres entiers
\item Vieillissement prématuré et accéléré
\item Lésions cutanées qui apparaissent et disparaissent sans explication
\item Douleurs nerveuses aiguës traversant le corps comme des éclairs
\item Crises convulsives de plus en plus fréquentes et sévères
\item Pourriture visible sur certaines parties du corps, généralement les extrémités
\item Écoulement noir des yeux, des oreilles ou du nez pendant la canalisation
\item Détérioration des organes internes conduisant à des défaillances multiples
\item Dans les stades terminaux, auto-combustion spontanée ou décomposition rapide du corps
\end{itemize}
\end{tenebresbox}

Bien que les femmes ne souffrent pas de la souillure, canaliser \textit{saidar} comporte néanmoins des risques significatifs :

\begin{pouvoirbox}{Contrecoups immédiats}
\begin{itemize}
\item Épuisement physique extrême pouvant mener à l'évanouissement ou au coma
\item Maux de tête débilitants qui peuvent persister pendant des jours
\item Nausées et vertiges intenses après un effort de canalisation prolongé
\item Sensibilité accrue à la lumière et aux sons, parfois pendant des semaines
\item Brûlures internes lorsque trop de Pouvoir est canalisé trop rapidement
\item Fièvres subites atteignant des températures dangereuses
\item Perte temporaire de la capacité à canaliser (quelques heures à plusieurs jours)
\item Perte de contrôle musculaire ou paralysie temporaire
\item Déconnexion sensorielle du monde environnant
\end{itemize}
\end{pouvoirbox}

\begin{pouvoirbox}{Risques à long terme}
\begin{itemize}
\item Dépendance croissante à la sensation d'euphorie procurée par le contact avec la \textit{saidar}
\item Détérioration progressive de la barrière mentale séparant la canalisatrice du flux du Pouvoir
\item ``Combustion'' -- une canalisatrice qui puise trop profondément peut littéralement s'incinérer de l'intérieur
\item Développement d'une incapacité à ``lâcher prise'' de la Source sans assistance
\item Apparition d'une aura visible permanente pour les autres canalisatrices, révélant constamment sa nature
\item Affaiblissement graduel des défenses naturelles contre les maladies communes
\item Vieillissement ralenti au point de devenir socialement aliénant (effet secondaire plutôt qu'un véritable danger)
\item Attrait irrésistible pour les objets liés au Pouvoir, menant parfois à des comportements à risque
\item ``Écho'' du Pouvoir -- capacité à sentir d'anciens tissages puissants, créant des distractions dangereuses
\end{itemize}
\end{pouvoirbox}

\begin{tenebresbox}{Évolution de l'Agenda du Ténébreux}
Si le dé des Ténèbres a été le plus haut, le Maître de Jeu peut remplacer une manifestation directe et immédiate par une manifestation indirecte, qui va se manifester plus tard. Voici quelques suggestion en fonction de la Marge de Réussite du Contrecoup :

\begin{description}
\item[1-2 Soubresaut] Malchance économique, diffusions de rumeurs, poisse météorologique
\item[3-4 Toile des Ténèbres] Création d'illusions, placement d'agents dormants, boost d'ennemis
\item[5-6 Cœur empoisonné] Manipulation des proches, mission pour des suppôts, zizanie dans le groupe
\item[7-8 Cible des Ténèbres] Manipulation du Motif. Cauchemars. Hommes-Gris, assassins.
\item[9-10 Ennemi des Ténèbres] Cible des Trollocs. Possession Maudite. Envoi d'un gholam.
\end{description}
\end{tenebresbox}

\subsection{Les utilisateurs du Pouvoir de l'Unique}

\subsubsection{La Tour Blanche}

La Tour Blanche à Tar Valon est le siège des \textit{Aes Sedai}, les utilisatrices les plus reconnues et organisées du Pouvoir de l'Unique. Leur influence s'étend à travers le monde, et les dirigeants de presque toutes les nations civilisées viennent à elles pour chercher conseil. À travers les siècles, Tar Valon a été un centre de connaissance et de pouvoir, mais aussi un foyer d'intrigues politiques.

Les \textit{Aes Sedai} sont liées par trois serments, prononcés sur le Bâton des Serments, un \textit{ter'angreal} de l'Âge des Légendes. Elles jurent de ne jamais mentir, de ne jamais fabriquer d'armes destinées à tuer des humains, et de ne jamais utiliser le Pouvoir comme une arme sauf contre les Ténébreux ou en dernière défense. Ces serments sont la base de leur autorité morale et politique.

Pour devenir \textit{Aes Sedai}, une femme doit suivre un apprentissage long et rigoureux. Elle commence comme novice, avant d'être éventuellement promue au rang d'Acceptée, pour finalement devenir une \textit{Aes Sedai} à part entière. Beaucoup ne complètent jamais cette formation, et certaines ne survivent pas à l'apprentissage du Pouvoir ou échouent lors de l'épreuve finale. Typiquement, un apprentissage dure de sept à dix ans, selon les aptitudes de l'étudiante.

Les \textit{Aes Sedai} sont divisées en sept Ajahs, chacun avec des intérêts et des objectifs spécifiques :

\begin{itemize}
\item L'Ajah Blanche se consacre à la logique et à la vérité
\item L'Ajah Grise aux négociations et à la médiation
\item L'Ajah Verte à la préparation pour la Dernière Bataille
\item L'Ajah Jaune à la guérison
\item L'Ajah Bleue à la justice et aux causes nobles
\item L'Ajah Brune à la connaissance et à l'érudition
\item L'Ajah Rouge à la traque des hommes capables de canaliser
\end{itemize}

Les représentants de chaque Ajah siègent au Hall de la Tour qui élit l'Amyrlin, la dirigeante des \textit{Aes Sedai}. L'Amyrlin sert à vie, bien que les Chroniques révèlent que certaines ont été déposées. Techniquement, l'Amyrlin détient un pouvoir absolu, mais en pratique, elle doit souvent négocier avec le Hall et les traditions anciennes.

\subsubsection{Autres traditions de canalisateurs}

En dehors de la Tour Blanche, d'autres traditions de canalisateurs existent à travers le monde :

\textbf{Les Sages Aiel} sont des femmes de force et de sagesse qui guident le peuple du Désert des Aiel. Elles utilisent le Pouvoir de l'Unique pour guérir et protéger, mais contrairement aux \textit{Aes Sedai}, elles ne sont pas organisées en Ajahs et ne prononcent pas les Trois Serments.

\textbf{Les Chercheuses de Vent} des Atha'an Miere, le Peuple de la Mer, utilisent le Pouvoir principalement pour contrôler les vents et les courants, permettant à leurs navires de naviguer même dans les conditions les plus difficiles.

\textbf{Les Seanchan} ont une vision très différente du Pouvoir. Dans leur empire au-delà de l'océan Arithien, les femmes capables de canaliser sont considérées comme dangereuses et sont contrôlées par des dispositifs appelés \textit{a'dam}. Ces femmes, appelées \textit{damane} (``celles qui sont en laisse''), sont utilisées comme armes vivantes et outils par leurs maîtresses, les \textit{sul'dam}.

\textbf{Les \textit{Asha'man}} représentent la tradition la plus récente. Fondée par le Dragon Réincarné lui-même, cette organisation d'hommes canalisateurs basée à la Tour Noire cherche à former des armes humaines pour la Dernière Bataille. Contrairement aux \textit{Aes Sedai}, les \textit{Asha'man} sont encouragés à utiliser le Pouvoir comme une arme.

\textbf{Les Rénégats}, \textit{Aes Sedai} de l'Âge des Légendes qui ont juré fidélité au Ténébreux, sont parmi les canalisateurs les plus puissants et les plus dangereux. Scellés avec leur maître pendant des millénaires, certains sont désormais libres et œuvrent dans l'ombre pour son retour.

\subsection{Conjuration}

A côté des canalisateurs et de la crainte qu'ils inspirent, de nombreux objets mythiques aux propriétés étonnantes traversent les Âges et sont convoités et documentés par les cercles de pouvoir.

Les \textit{ter'angreal}, \textit{angreal} et \textit{sa'angreal} sont des artéfacts de l'Âge des Légendes permettant d'interagir avec le Pouvoir de l'Unique de façons diverses.

Les \textit{ter'angreal} sont des objets créés pour accomplir une fonction spécifique en utilisant le Pouvoir, comme le Bâton des Serments qui lie les \textit{Aes Sedai} à leurs vœux.

Les \textit{angreal} et \textit{sa'angreal} amplifient la quantité de Pouvoir qu'un canalisateur peut manier en toute sécurité, les \textit{sa'angreal} étant significativement plus puissants. Ces objets sont généralement liés à un sexe spécifique -- un \textit{angreal} conçu pour un homme ne peut être utilisé par une femme, et vice versa.

La plupart de ces objets exceptionnels viennent de l'Âge des Légendes car ils ont nécessité des rituels qui mélangeaient le \textit{saidin} et la \textit{saidar}. Aujourd'hui, un personnage ou un groupe de personnage pourrait certainement difficilement conjurer un objet magique de légende. Néanmoins, en suivant les règles d'\textsc{Engrenages}, un cercle de pouvoir qui arrive à une MR de 10 avec une compétence de 10 (puisque la MR est limitée par la compétence) peut créer un effet permanent dans un objet. Si le tissage intégré dans l'objet requiert plusieurs Éléments, il faut soit que le meneur ait plusieurs compétences à 10, soit que parmi le cercle, au moins un canalisateur ait l'élément demandé à 10.
