\section{Personnages Prêts à Jouer}

\subsection{Introduction : Un Groupe Forgé par l'Exil}

Ces six personnages forment un groupe cohérent d'aventuriers unis par un fil commun : tous ont fui leur vie d'origine pour échapper à des situations impossibles. Leurs chemins se sont croisés sur les routes des Terres de l'Ouest, créant des liens forgés par la nécessité et renforcés par l'entraide mutuelle.

\textbf{Thèmes du groupe :}
\begin{itemize}
    \item \textbf{Rédemption et seconde chance} : Chacun cherche à se racheter d'un passé douloureux
    \item \textbf{Solidarité des marginaux} : Exclus de la société ``normale'', ils se créent leur propre famille
    \item \textbf{Quête de connaissance} : Recherche active de solutions à leurs problèmes (notamment la folie de Fernando)
    \item \textbf{Équilibre des cultures} : Représentation diverse des peuples des Terres de l'Ouest
\end{itemize}

\vspace{0.5cm}
\noindent\rule{\textwidth}{0.4pt}
\vspace{0.5cm}

\subsection{Fael Barsere --- Ex-Enfant de la Lumière}
\textit{Leader du groupe, guérisseur itinérant}

\noindent\textbf{Origine :} Ghealdan\\
\textbf{Âge :} 32 ans\\
\textbf{Profession :} Guérisseur sans Le Pouvoir, ancien militaire

\subsubsection{Histoire Personnelle}

Ancien Enfant de la Lumière, Fael a déserté après avoir assisté aux excès des Questionneurs lors de massacres qu'il ne pouvait plus cautionner. Sa fuite l'a mené à rencontrer Alise, une sage-femme du Kin, dont les méthodes de guérison lui ont ouvert les yeux sur une voie différente. Elle lui a enseigné les bases de la médecine traditionnelle, et il a décidé de consacrer sa vie à soigner plutôt qu'à faire du mal.

\subsubsection{Contexte Culturel --- Ghealdan}

Le Ghealdan est une nation instable, constamment troublée par les prétendants au titre de Dragon Réincarné et les conflits internes. Les Enfants de la Lumière y recrutent activement, prêchant la purification du monde par l'épée et la discipline. Cette organisation militaro-religieuse forme des guerriers fanatiques convaincus que tous les maux viennent de l'Ombre et du Pouvoir Unique.

\subsubsection{Traits de Roleplay}

\begin{itemize}
    \item \textbf{Rédemption par les soins} : Compense ses erreurs passées en soignant quiconque a besoin d'aide
    \item \textbf{Méfiance instinctive} : Son passé le rend naturellement suspicieux, mais cette méfiance l'aide à protéger Fernando
    \item \textbf{Leader discret} : Préfère voyager en groupe, propose spontanément d'accompagner ceux qui en ont besoin
    \item \textbf{Apparence trompeuse} : Son allure d'ancien militaire rebute avant qu'on découvre sa vraie nature
\end{itemize}

\subsubsection{Habitudes et Manies}

\begin{itemize}
    \item Vérifie instinctivement les sorties dans chaque lieu
    \item Examine automatiquement l'état de santé des gens qu'il rencontre
    \item Garde ses armes en parfait état par habitude militaire
    \item Se lève toujours avant l'aube
\end{itemize}

\subsubsection{Traits (2 points)}

\begin{itemize}
    \item \textbf{Ancien Enfant de la Lumière} (1) : Connaissance des tactiques militaires, méfiance instinctive
    \item \textbf{Guérisseur de renom} (1) : Réputation grandissante de thérapeute efficace
\end{itemize}

\subsubsection{Destinée} \textit{(à déterminer par le MJ)}

Exemples : ``Celui qui guérira la blessure entre Lumière et Ténèbres'' ou ``Le pont entre les traditions ennemies''

\subsubsection{Vie d'autrefois}

``Soldat fanatique au service d'une cause qu'il croyait juste''

\subsubsection{Compétences (25 points)}

\noindent\textbf{Physiques (8 points)}
\begin{itemize}
    \item Combat : 3 \textit{(Spécialisation : Épée à 5+)}
    \item Athlétisme : 3
    \item Animaux : 2
\end{itemize}

\noindent\textbf{Mentales (10 points)}
\begin{itemize}
    \item Rural : 5 \textit{(Spécialisation : Herboristerie à 5+)}
    \item Folklore : 3
    \item Érudition : 2
\end{itemize}

\noindent\textbf{Sociales (7 points)}
\begin{itemize}
    \item Faconde : 4
    \item Représentation : 2
    \item Argutie : 1
\end{itemize}

\noindent\textbf{Occultes}
\begin{itemize}
    \item Aucune (ne peut pas canaliser)
\end{itemize}

\subsubsection{Relations dans le Groupe}

\begin{itemize}
    \item \textbf{Avec Fernando} : Relation de confiance mutuelle basée sur l'expérience commune de la fuite
    \item \textbf{Avec Alise} : Reconnaissance profonde envers celle qui lui a montré une autre voie
    \item \textbf{Avec Tamira} : Fierté discrète d'avoir sauvé une guerrière d'honneur
    \item \textbf{Avec Theron} : Méfiance polie envers les nobles, même déchus
    \item \textbf{Avec Bethamin} : Curiosité prudente pour ses méthodes étranges mais efficaces
\end{itemize}

\vspace{0.5cm}
\noindent\rule{\textwidth}{0.4pt}
\vspace{0.5cm}

\subsection{Fernando Almandre --- Canalisateur masculin}
\textit{Noble déchu en fuite, érudit tourmenté}

\noindent\textbf{Origine :} Amadicia (famille noble)\\
\textbf{Âge :} 28 ans\\
\textbf{Profession :} Fugitif, ancien érudit

\subsubsection{Histoire Personnelle}

Issu d'une famille noble d'Amadicia, Fernando a fui après un événement qui aurait pu révéler ses capacités de canalisation. Sa quête principale est de trouver un moyen de continuer à vivre sans devenir fou et dangereux pour ses proches. Ses symptômes restent légers grâce à sa discipline personnelle et à l'aide combinée de Fael et Bethamin.

\subsubsection{Contexte Culturel --- Amadicia}

L'Amadicia est une théocratie dirigée par les Enfants de la Lumière, où Le Pouvoir Unique est absolument proscrit et craint. Les nobles y vivent dans la terreur constante d'être dénoncés pour hérésie. Fernando a grandi dans ce carcan de conventions rigides et de surveillance constante, développant les réflexes de dissimulation et la culture raffinée de la noblesse.

\subsubsection{Traits de Roleplay}

\begin{itemize}
    \item \textbf{Noble déchu} : Éducation raffinée mais plus de protection sociale
    \item \textbf{Lutte quotidienne} : Combat constamment contre les premiers signes de la folie
    \item \textbf{Savoir dangereux} : Possède des connaissances qu'il ne peut révéler sans risque
    \item \textbf{Espoir têtu} : Refuse d'accepter que la folie soit inéluctable
\end{itemize}

\subsubsection{Habitudes et Manies}

\begin{itemize}
    \item Médite chaque matin pour maintenir son équilibre mental
    \item Évite de toucher Le Pouvoir sauf en cas d'absolue nécessité
    \item Prend des notes codées sur ses découvertes
    \item Tressaille parfois sans raison (premiers signes de la folie)
\end{itemize}

\subsubsection{Traits (2 points)}

\begin{itemize}
    \item \textbf{Éducation noble} (1) : Connaissances étendues, manières raffinées
    \item \textbf{Résistance à la Souillure} (1) : Progression ralentie de la folie grâce à sa discipline
\end{itemize}

\subsubsection{Destinée} \textit{(à déterminer par le MJ)}

``L'homme qui trouvera une nouvelle voie pour les canalisateurs masculins''

\subsubsection{Vie d'autrefois}

``Érudit noble destiné à hériter des terres familiales''

\subsubsection{Compétences (25 points)}

\noindent\textbf{Physiques (4 points)}
\begin{itemize}
    \item Combat : 2
    \item Dépassement : 2
\end{itemize}

\noindent\textbf{Mentales (11 points)}
\begin{itemize}
    \item Érudition : 3
    \item Folklore : 4
    \item Thaumatologie : 2
    \item Citadin : 2
\end{itemize}

\noindent\textbf{Sociales (5 points)}
\begin{itemize}
    \item Faconde : 3
    \item Argutie : 2
\end{itemize}

\noindent\textbf{Occultes (5 points)}
\begin{itemize}
    \item Terre : 2 \textit{(Spécialisation : Sculpture à 5+)}
    \item Feu : 2
    \item Esprit : 1
\end{itemize}

\subsubsection{Relations dans le Groupe}

\begin{itemize}
    \item \textbf{Avec Fael} : Amitié profonde née de la compréhension mutuelle
    \item \textbf{Avec Bethamin} : Reconnaissance pour son aide unique contre la folie
    \item \textbf{Avec Alise} : Respect pour sa sagesse pratique
    \item \textbf{Avec Theron} : Inquiétude que ses chansons n'attirent l'attention
    \item \textbf{Avec Tamira} : Appréciation de sa franchise directe
\end{itemize}

\vspace{0.5cm}
\noindent\rule{\textwidth}{0.4pt}
\vspace{0.5cm}

\subsection{Alise Brennan --- Femme du Kin}
\textit{Sage-femme discrète, mentor du groupe}

\noindent\textbf{Origine :} Frontière Ghealdan/Altara\\
\textbf{Âge :} Apparence de 45 ans (en réalité 70+)\\
\textbf{Profession :} Sage-femme, guérisseuse

\subsubsection{Histoire Personnelle}

Ancienne novice de la Tour Blanche, renvoyée après une épreuve ratée. Plus âgée qu'elle n'en a l'air grâce au ralentissement du vieillissement lié au Pouvoir, seules les Aes Sedai les plus anciennes pourraient se souvenir d'elle. Elle a trouvé refuge dans le Kin et pratique discrètement la guérison. C'est elle qui a ouvert les yeux de Fael sur une voie différente de la violence.

\subsubsection{Contexte Culturel --- Le Kin}

Le Kin est une organisation secrète de femmes capables de canaliser mais qui ont échoué à devenir Aes Sedai ou ont fui la Tour Blanche. Elles vivent cachées dans la société normale, utilisant discrètement leurs capacités pour aider sans jamais révéler leur vraie nature. Cette clandestinité crée des femmes prudentes, observatrices et expertes dans l'art de passer inaperçues.

\subsubsection{Traits de Roleplay}

\begin{itemize}
    \item \textbf{Sagesse pratique} : Approche terre-à-terre, peu encline aux dogmes
    \item \textbf{Mentor discret} : Guide sans imposer, enseigne par l'exemple
    \item \textbf{Réseau secret} : Contacts dans le Kin à travers le continent
    \item \textbf{Expérience cachée} : Plus d'années et de sagesse qu'elle n'en montre
\end{itemize}

\subsubsection{Habitudes et Manies}

\begin{itemize}
    \item Hume l'air pour détecter les maladies
    \item Collecte automatiquement les herbes utiles en voyageant
    \item Observe les gens avec un regard perçant mais bienveillant
    \item Cache instinctivement sa vraie connaissance du Pouvoir
\end{itemize}

\subsubsection{Traits (2 points)}

\begin{itemize}
    \item \textbf{Membre du Kin} (2) : Réseau secret, connaissances sur les canalisatrices cachées
\end{itemize}

\subsubsection{Destinée} \textit{(à déterminer par le MJ)}

``La gardienne des savoirs perdus du Kin''

\subsubsection{Vie d'autrefois}

``Novice pleine d'espoir rêvant de devenir Aes Sedai''

\subsubsection{Compétences (25 points)}

\noindent\textbf{Physiques (5 points)}
\begin{itemize}
    \item Athlétisme : 3
    \item Animaux : 2
\end{itemize}

\noindent\textbf{Mentales (10 points)}
\begin{itemize}
    \item Rural : 5 \textit{(Spécialisation : Guérison traditionnelle à 5+)}
    \item Érudition : 3
    \item Folklore : 2
\end{itemize}

\noindent\textbf{Sociales (6 points)}
\begin{itemize}
    \item Faconde : 3
    \item Argutie : 2
    \item Maraude : 1
\end{itemize}

\noindent\textbf{Occultes (4 points)}
\begin{itemize}
    \item Eau : 2 \textit{(Spécialisation : Guérison à 5+)}
    \item Esprit : 2
\end{itemize}

\subsubsection{Relations dans le Groupe}

\begin{itemize}
    \item \textbf{Avec Fael} : Fierté maternelle discrète de son évolution
    \item \textbf{Avec Fernando} : Compassion pour sa lutte, respect pour son courage
    \item \textbf{Avec Bethamin} : Curiosité pour ses méthodes étrangères
    \item \textbf{Avec Theron} : Amusement devant ses tentatives de charme
    \item \textbf{Avec Tamira} : Appréciation de son honnêteté directe
\end{itemize}

\vspace{0.5cm}
\noindent\rule{\textwidth}{0.4pt}
\vspace{0.5cm}

\subsection{Theron Chuliandred --- Trouvère noble}
\textit{Ménestrel exilé, chroniqueur du groupe}

\noindent\textbf{Origine :} Cairhien (maison noble mineure)\\
\textbf{Âge :} 30 ans\\
\textbf{Profession :} Ménestrel, chroniqueur

\subsubsection{Histoire Personnelle}

Exilé de Cairhien après s'être fâché avec un membre influent de la cour, Theron voyage maintenant en observant le monde réel plutôt que les intrigues de palais. Il est particulièrement fasciné par l'amitié improbable entre Fael et Fernando, y voyant le sujet d'une chanson sur ceux qui brisent les étiquettes que la société leur impose.

\subsubsection{Contexte Culturel --- Cairhien}

Cairhien est la patrie du ``Grand Jeu'', un système complexe d'intrigues politiques où chaque geste peut cacher une signification. Les nobles cairhienins excellent dans la manipulation subtile et les alliances temporaires. Theron a grandi dans cet environnement de méfiance raffinée, développant un sens aigu de l'observation et une fascination pour les motivations humaines authentiques.

\subsubsection{Traits de Roleplay}

\begin{itemize}
    \item \textbf{Observateur avisé} : Analyse les relations et motivations avec un œil d'artiste
    \item \textbf{Ancien courtisan} : Maîtrise les codes sociaux mais s'en lasse
    \item \textbf{Chroniqueur passionné} : Transforme la réalité en récits mémorables
    \item \textbf{Exilé philosophe} : Trouve plus de vérité sur la route qu'à la cour
\end{itemize}

\subsubsection{Habitudes et Manies}

\begin{itemize}
    \item Compose mentalement des vers sur les situations qu'il observe
    \item Ajuste inconsciemment sa manière de parler selon son interlocuteur
    \item Prend des notes discrètes pour ses futures chansons
    \item Teste ses histoires sur le groupe avant de les présenter ailleurs
\end{itemize}

\subsubsection{Traits (2 points)}

\begin{itemize}
    \item \textbf{Maître du Grand Jeu} (1) : Expertise en manipulation sociale et lecture des intentions
    \item \textbf{Ménestrel renommé} (1) : Talents artistiques reconnus, facilite les contacts
\end{itemize}

\subsubsection{Destinée} \textit{(à déterminer par le MJ)}

``Le chroniqueur qui immortalisera les héros oubliés''

\subsubsection{Vie d'autrefois}

``Noble courtisan promis à une carrière brillante dans les intrigues''

\subsubsection{Compétences (25 points)}

\noindent\textbf{Physiques (4 points)}
\begin{itemize}
    \item Athlétisme : 2
    \item Animaux : 2
\end{itemize}

\noindent\textbf{Mentales (8 points)}
\begin{itemize}
    \item Folklore : 3
    \item Citadin : 3
    \item Érudition : 2
\end{itemize}

\noindent\textbf{Sociales (13 points)}
\begin{itemize}
    \item Représentation : 5 \textit{(Spécialisation : Musique/Chant à 5+)}
    \item Faconde : 4
    \item Argutie : 2
    \item Maraude : 2
\end{itemize}

\noindent\textbf{Occultes}
\begin{itemize}
    \item Aucune
\end{itemize}

\subsubsection{Relations dans le Groupe}

\begin{itemize}
    \item \textbf{Avec Fael} : Fascination pour sa transformation personnelle
    \item \textbf{Avec Fernando} : Intrigué par sa lutte héroïque contre le destin
    \item \textbf{Avec Alise} : Respect pour sa sagesse cachée
    \item \textbf{Avec Bethamin} : Curiosité pour cette culture si différente
    \item \textbf{Avec Tamira} : Admiration pour sa simplicité noble
\end{itemize}

\vspace{0.5cm}
\noindent\rule{\textwidth}{0.4pt}
\vspace{0.5cm}

\subsection{Bethamin Kettara --- \textit{Der'sul'dam}}
\textit{Ancienne dresseuse seanchane, archéologue amateur}

\noindent\textbf{Origine :} Empire Seanchan\\
\textbf{Âge :} 26 ans\\
\textbf{Profession :} Ancienne dresseuse, exilée, archéologue amateur

\subsubsection{Histoire Personnelle}

Ancienne \textit{sul'dam} qui a découvert ses capacités de canalisation pendant son service, ce qui a provoqué sa fuite de l'Empire. Rongée par la culpabilité d'avoir contrôlé des \textit{damane}, elle aide Fernando à maîtriser sa folie en adaptant ses connaissances du contrôle des canalisatrices. Cette quête la pousse vers l'archéologie de l'Âge des Légendes.

\subsubsection{Contexte Culturel --- Empire Seanchan}

L'Empire Seanchan repose sur une hiérarchie sociale rigide où chacun connaît exactement sa place. Les \textit{sul'dam} occupent une position respectée car elles contrôlent les femmes capables de canaliser réduites en esclavage. Cette culture valorise l'ordre et l'obéissance absolue. Bethamin a grandi dans cette certitude morale, ce qui rend sa prise de conscience actuelle d'autant plus dévastatrice.

\subsubsection{Traits de Roleplay}

\begin{itemize}
    \item \textbf{Culpabilité profonde} : Hantée par ses actes passés envers les \textit{damane}
    \item \textbf{Expertise unique} : Connaît intimement les techniques de contrôle de canalisation
    \item \textbf{Étrangère absolue} : Culture totalement différente, codes sociaux décalés
    \item \textbf{Chercheuse obsessionnelle} : Poursuit les artefacts anciens avec détermination
\end{itemize}

\subsubsection{Habitudes et Manies}

\begin{itemize}
    \item Utilise parfois des expressions seanchanes sans s'en rendre compte
    \item Évalue instinctivement la force de canalisation des femmes rencontrées
    \item Dessine des diagrammes de ses théories sur les objets anciens
    \item Sursaute encore quand on la touche sans prévenir
\end{itemize}

\subsubsection{Traits (2 points)}

\begin{itemize}
    \item \textbf{Ancienne \textit{sul'dam}} (1) : Expertise dans le contrôle des canalisateurs
    \item \textbf{Fugitive de l'Empire} (1) : Recherchée, mais connaissance unique des Seanchans
\end{itemize}

\subsubsection{Destinée} \textit{(à déterminer par le MJ)}

``Celle qui révélera les secrets de l'Âge des Légendes''

\subsubsection{Vie d'autrefois}

``Dresseuse respectée au service de l'Empire éternel''

\subsubsection{Compétences (25 points)}

\noindent\textbf{Physiques (6 points)}
\begin{itemize}
    \item Combat : 4
    \item Athlétisme : 2
\end{itemize}

\noindent\textbf{Mentales (12 points)}
\begin{itemize}
    \item Thaumatologie : 5 \textit{(Spécialisation : Contrôle des canalisateurs à 5+)}
    \item Érudition : 4
    \item Folklore : 3
\end{itemize}

\noindent\textbf{Sociales (3 points)}
\begin{itemize}
    \item Représentation : 2
    \item Argutie : 1
\end{itemize}

\noindent\textbf{Occultes (4 points)}
\begin{itemize}
    \item Air : 2
    \item Esprit : 2
\end{itemize}

\subsubsection{Relations dans le Groupe}

\begin{itemize}
    \item \textbf{Avec Fernando} : Détermination farouche à l'aider, forme de rédemption personnelle
    \item \textbf{Avec Fael} : Respect pour sa transformation similaire à la sienne
    \item \textbf{Avec Alise} : Apprentissage mutuel de techniques différentes
    \item \textbf{Avec Theron} : Incompréhension des subtilités sociales continentales
    \item \textbf{Avec Tamira} : Discussions passionnées sur les légendes et objets anciens
\end{itemize}

\vspace{0.5cm}
\noindent\rule{\textwidth}{0.4pt}
\vspace{0.5cm}

\subsection{Tamira Kodara --- Chasseuse du Shienar}
\textit{Guerrière frontalière, protectrice loyale}

\noindent\textbf{Origine :} Shienar (garde frontalière)\\
\textbf{Âge :} 29 ans\\
\textbf{Profession :} Guerrière, traqueur

\subsubsection{Histoire Personnelle}

Guerrière du Shienar en mission longue, blessée par une créature du Ténébreux et soignée par Fael. Son éloignement de la frontière lui donne l'occasion de découvrir le monde au-delà de sa terre natale. Elle a contracté une dette d'honneur envers Fael et le suit par loyauté, tout en étant fascinée par les discussions de Bethamin sur les légendes anciennes.

\subsubsection{Contexte Culturel --- Shienar}

Le Shienar est une nation forgée par la guerre constante contre l'Ombre à la Flétrissure. Cette lutte permanente a créé une société guerrière où l'honneur, le courage et la loyauté sont les valeurs suprêmes. Les Shienarians développent une franchise directe et un sens du devoir absolu. Tamira a grandi dans cette culture de l'honneur martial où la dette de vie se paie par la loyauté éternelle.

\subsubsection{Traits de Roleplay}

\begin{itemize}
    \item \textbf{Code d'honneur absolu} : Loyauté inébranlable envers ceux qui l'ont aidée
    \item \textbf{Guerrière née} : Combat et survie sont ses domaines naturels
    \item \textbf{Curiosité nouvelle} : Découvre un monde plus vaste que la Flétrissure
    \item \textbf{Franchise directe} : Dit ce qu'elle pense sans détours
\end{itemize}

\subsubsection{Habitudes et Manies}

\begin{itemize}
    \item Vérifie ses armes chaque soir par routine militaire
    \item Évalue automatiquement les menaces dans chaque environnement
    \item Pose des questions directes sur les sujets qui l'intriguent
    \item Entretient méticuleusement son équipement
\end{itemize}

\subsubsection{Traits (2 points)}

\begin{itemize}
    \item \textbf{Guerrière du Shienar} (1) : Formation militaire, résistance aux créatures de l'Ombre
    \item \textbf{Dette d'honneur} (1) : Loyauté absolue envers Fael, mais contraintes morales
\end{itemize}

\subsubsection{Destinée} \textit{(à déterminer par le MJ)}

``La gardienne qui protégera les innocents loin de sa terre natale''

\subsubsection{Vie d'autrefois}

``Soldate dévouée à la protection de la frontière maudite''

\subsubsection{Compétences (25 points)}

\noindent\textbf{Physiques (15 points)}
\begin{itemize}
    \item Combat : 5 \textit{(Spécialisation : Arc à 5+)}
    \item Athlétisme : 5 \textit{(Spécialisation : Survie à 5+)}
    \item Dépassement : 3
    \item Animaux : 2
\end{itemize}

\noindent\textbf{Mentales (6 points)}
\begin{itemize}
    \item Rural : 3
    \item Folklore : 3
\end{itemize}

\noindent\textbf{Sociales (4 points)}
\begin{itemize}
    \item Représentation : 2
    \item Argutie : 2
\end{itemize}

\noindent\textbf{Occultes}
\begin{itemize}
    \item Aucune
\end{itemize}

\subsubsection{Relations dans le Groupe}

\begin{itemize}
    \item \textbf{Avec Fael} : Dette d'honneur absolue, respect profond
    \item \textbf{Avec Fernando} : Protection instinctive du ``faible'' du groupe
    \item \textbf{Avec Alise} : Appréciation de sa sagesse pratique
    \item \textbf{Avec Theron} : Amusement devant ses manières courtoises
    \item \textbf{Avec Bethamin} : Fascination pour ses histoires de héros et légendes
\end{itemize}

\vspace{0.5cm}
\noindent\rule{\textwidth}{0.4pt}
\vspace{0.5cm}

\subsection{Dynamique de Groupe}

\subsubsection{Formation et Leadership}

Le groupe s'est formé progressivement, au hasard des rencontres sur la route. Fael, par son passé militaire et sa nature protectrice, a naturellement pris le rôle de leader en proposant à chaque nouveau membre de les accompagner. Malgré les difficultés que représente un groupe aussi visible, il préfère voyager accompagné.

\subsubsection{Objectifs Communs}

\begin{itemize}
    \item \textbf{Quête principale} : Aider Fernando à trouver un moyen de contrôler sa folie
    \item \textbf{Recherches} : Explorer les traces de l'Âge des Légendes (influence de Bethamin)
    \item \textbf{Survie} : Éviter les autorités tout en aidant ceux qu'ils croisent
\end{itemize}

\subsubsection{Défis du Groupe}

\begin{itemize}
    \item \textbf{Visibilité} : Un homme canalisant et une Seanchane sont difficiles à cacher
    \item \textbf{Cultures différentes} : Codes d'honneur et habitudes sociales parfois incompatibles
    \item \textbf{Secrets} : Chacun cache une partie de son passé
\end{itemize}

\subsubsection{Habitudes de Voyage}

\begin{itemize}
    \item Évitent les grandes villes et routes principales
    \item Fael soigne les villageois en échange du gîte et du couvert
    \item Theron divertit dans les auberges pour gagner quelques pièces
    \item Bethamin et Fernando étudient discrètement les ruines et artefacts rencontrés
\end{itemize}

\subsubsection{Hooks Narratifs pour le MJ}

\begin{enumerate}
    \item \textbf{Le passé rattrape Fael} : D'anciens ``frères'' Enfants de la Lumière le retrouvent
    \item \textbf{La folie progresse} : Fernando doit faire face à une crise majeure
    \item \textbf{Contacts du Kin} : Alise reçoit un message urgent de ses sœurs
    \item \textbf{Le Grand Jeu continue} : Les intrigues cairhiennes rattrapent Theron
    \item \textbf{Chasseurs seanchans} : L'Empire envoie des agents récupérer Bethamin
    \item \textbf{Appel du devoir} : Le Shienar rappelle Tamira pour la guerre contre l'Ombre
\end{enumerate}

\vspace{0.5cm}
\noindent\rule{\textwidth}{0.4pt}
\vspace{0.5cm}

\noindent\textit{Ces personnages sont conçus pour fonctionner ensemble tout en offrant de multiples angles narratifs. Chacun apporte des compétences uniques et des complications personnelles qui peuvent alimenter des sessions entières de jeu de rôle riche et nuancé.}
