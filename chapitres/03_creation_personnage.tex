\subsection{Étapes de Création}

Créer un personnage suit sept étapes simples. Choisissez d'abord votre concept général. Définissez ensuite votre origine culturelle. Répartissez vos Compétences selon vos priorités. Calculez vos Jauges de base. Ajoutez vos Pouvoirs spéciaux si applicable. Équipez votre personnage. Établissez ses relations.

\subsubsection{Concept de Base}

Votre personnage a une profession, une origine et une motivation. L'\textit{Aes Sedai} cherche la connaissance. Le marchand accumule les richesses. Le garde protège son seigneur. Cette base guide tous vos autres choix.

\subsubsection{Attribution des Points}

La création d'un personnage dans La Roue du Temps suit les étapes classiques d'un personnage d'Engrenages :
\begin{enumerate}
\item Répartissez 2 points de Trait (soit deux Traits à 1 ou un Trait à 2). Le personnage peut disposer d'un troisième point de Trait si ce point est compensé par un Trait défavorable. Pour savoir si un Trait est défavorable, il suffit de savoir si vous, en tant que MJ ou joueur, aimeriez que ce Trait vous décrive.
\item Distribuez 25 points dans les diverses Compétences.
\item Choisissez un nom à votre personnage.
\item Déterminez l'équipement de votre personnage. Il n'existe aucune règle stricte pour déterminer le matériel. L'équipement est fixé, avec l'accord du MJ, en fonction du background décrit et des envies du joueur. De manière générale, les personnages disposant d'un niveau de Compétence supérieur à 5 possèdent le nécessaire pour la pratiquer.
\end{enumerate}

\subsection{Background et Traits}

Le background de votre personnage va ressortir à travers les Traits que vous allez choisir. Voici une liste de traits liés à votre nation d'origine.

\subsection{Origines Culturelles}

Chaque nation forme ses enfants différemment. Ces différences se traduisent par des \textbf{Traits culturels gratuits} (qui ne comptent pas dans vos 2-3 points de Trait) et des Compétences favorisées. Choisissez votre origine selon votre concept et sélectionnez 1-2 Traits culturels appropriés.

\subsubsection{Andor}

Les plaines d'Andor élèvent des nobles et des soldats. L'éducation courtoise côtoie la tradition militaire. Les Maisons nobles tissent des réseaux complexes.

\textbf{Traits disponibles :}
\begin{itemize}
\item Rompu aux intrigues
\item Tradition militaire
\item Éducation rigoureuse
\end{itemize}

\textbf{Compétences favorisées :} Érudition, Argutie, Combat, Citadin

\textit{Un personnage avec Éducation noble (2) gagne +2 aux tests d'étiquette et de généalogie.}

\subsubsection{Cairhien}

Les tours de Cairhien abritent les maîtres du Grand Jeu. Chaque geste cache une intention. L'intrigue politique règne en art suprême. Le commerce finance ces ambitions.

\textbf{Traits disponibles :}
\begin{itemize}
\item Maître du Grand Jeu
\item Commerçant né
\item Réservé
\end{itemize}

\textbf{Compétences favorisées :} Argutie, Citadin, Érudition, Maraude

\textit{Maître du Grand Jeu (3) détecte automatiquement les manipulations sociales et révèle les motivations cachées.}

\subsubsection{Tear}

La Pierre de Tear domine l'embouchure d'Erinin. Les Hauts Seigneurs gouvernent par tradition. Une méfiance ancestrale oppose cette nation aux \textit{Aes Sedai}. La mer nourrit sa prospérité.

\textbf{Traits disponibles :}
\begin{itemize}
\item Sang de pêcheur (1-2)
\item Méfiance des \textit{Aes Sedai} (1-3)
\item Fierté tearienne (1-2)
\end{itemize}

\textbf{Compétences favorisées :} Mer, Combat, Rural, Dépassement

\textit{Méfiance des \textit{Aes Sedai} (2) donne +2 contre la Compulsion et les influences mentales magiques.}

\subsubsection{Saldaea}

Les Marches forgent les meilleurs cavaliers. La frontière avec la Flétrissure trempe les caractères. Le tempérament ardent s'allie au courage. L'Ombre teste constamment leurs défenses.

\textbf{Traits disponibles :}
\begin{itemize}
\item Cavalier des Marches (1-3)
\item Lame des Frontières (1-2)
\item Tempérament de feu (1-2)
\end{itemize}

\textbf{Compétences favorisées :} Animaux, Combat, Dépassement, Rural

\textit{Cavalier des Marches (3) permet le combat monté sans pénalités et des manœuvres équestres spectaculaires.}

\paragraph{\textsc{Illian}}
\textit{République maritime aux traditions de Compagnie}

\textbf{Traits culturels disponibles :}
\begin{itemize}
\item \textbf{Compagnon d'Illian} (1-3) : Membre des Compagnies mercenaires
\item \textbf{Accent reconnaissable} (1) : Particularité linguistique distinctive
\item \textbf{Tradition navale} (1-2) : Expertise maritime et commerciale
\end{itemize}

\textbf{Compétences favorisées :} Mer, Combat, Citadin, Faconde

\paragraph{\textsc{Arad Doman}}
\textit{Nation des négociants et séducteurs}

\textbf{Traits culturels disponibles :}
\begin{itemize}
\item \textbf{Art de la séduction} (1-3) : Maîtrise de la persuasion et du charme
\item \textbf{Négociant né} (1-2) : Intuition commerciale et réseau marchand
\item \textbf{Excessif} (1) : Tendance aux comportements extrêmes
\end{itemize}

\textbf{Compétences favorisées :} Faconde, Argutie, Citadin, Représentation

\subsubsection{Peuples Spéciaux}

\paragraph{\textsc{Aiels}}
\textit{Peuple du désert aux codes d'honneur stricts}

\textbf{Traits obligatoires :}
\begin{itemize}
\item \textbf{\textit{Ji'e'toh}} (2-3) : Code d'honneur complexe et inflexible
\item \textbf{Guerrier du désert} (1-3) : Adaptation parfaite à l'environnement aride
\item \textbf{Voile de guerre} (1-2) : Capacités martiales exceptionnelles
\end{itemize}

\textbf{Restrictions importantes :}
\begin{itemize}
\item Interdiction de porter l'épée (sauf Clan-Chefs)
\item Obligation de couvrir le visage en combat
\item Code moral contraignant les actions
\end{itemize}

\textbf{Compétences favorisées :} Combat, Dépassement, Rural, Animaux

\textit{Application technique :} Le trait ``\textit{Ji'e'toh} (3)'' impose des restrictions comportementales strictes mais accorde des bonus en situation d'honneur et de courage.

\paragraph{\textsc{Ogiers}}
\textit{Race ancienne aux capacités architecturales}

\textbf{Traits raciaux :}
\begin{itemize}
\item \textbf{Longévité} (3) : Espérance de vie de plusieurs siècles
\item \textbf{Force pacifique} (2-3) : Puissance physique et tempérament calme
\item \textbf{Constructeur né} (2-3) : Intuition architecturale surnaturelle
\end{itemize}

\textbf{Limitations :}
\begin{itemize}
\item \textbf{Mal du Longing} : Besoin de retourner régulièrement aux \textit{stedding}
\item Difficulté avec les changements rapides
\item Répugnance naturelle à la violence
\end{itemize}

\textbf{Compétences favorisées :} Créativité, Érudition, Dépassement, Thaumatologie

\subsection{Langues}

Le nombre de langues différentes et de dialectes parlés dans le monde de La Roue du temps est important. Pour refléter le fait que les polyglottes y sont très nombreux, un personnage parle sa langue native, mais aussi un nombre de langues égal à Érudition / 2 (arrondi à l'inférieur).

\subsection{\textit{Ta'veren} : Système des Attaches (Règle Optionnelle)}

\textit{Cette section utilise les règles avancées ``Attaches et Destin'' d'Engrenages (voir Guide du MJ, Chapitre 7)}

\textbf{Note importante :} Seuls les personnages explicitement \textit{Ta'veren} utilisent ces règles. La plupart des PJ sont des ``Témoins du Dessin'' ou ``Liés à un \textit{Ta'veren}''.

\subsubsection{Création d'un \textit{Ta'veren}}

\textbf{Prérequis :} Accord explicite du MJ (extrêmement rare)

\textbf{Attaches Obligatoires (6 points total) :}

\begin{tabular}{|l|c|p{8cm}|}
\hline
\textbf{Attache} & \textbf{Valeur} & \textbf{Description} \\
\hline
\textbf{Vie d'Autrefois} & 3 & Famille, foyer, rêves simples d'avant l'aventure \\
\hline
\textbf{Humanité} & 2 & Compassion, capacité aux relations normales \\
\hline
\textbf{Innocence} & 1 & Vision naïve du monde, confiance naturelle \\
\hline
\end{tabular}

\textit{Exemples de Vie d'Autrefois : ``Berger des Deux-Rivières +3'', ``Sage-femme du village +3'', ``Noble promis à gouverner +3''}

\subsubsection{Mécanique des Attaches}

\textbf{Sacrifice Volontaire :}
\begin{itemize}
\item \textbf{Bonus au test :} +X (= valeur de l'attache)
\item \textbf{Perte permanente :} L'attache diminue de 1 point
\item \textbf{Déclenchement :} Manifestation du Dessin (voir Guide du MJ)
\end{itemize}

\textbf{Récupération des Jauges :}
\begin{itemize}
\item \textbf{Rythme ralenti :} 3 points par scénario (au lieu de 3 par nuit)
\item \textbf{Récupération complète :} Sacrifice 1 point d'Attache = retrouver 10 points de jauge
\end{itemize}

\textbf{Exemple de Dilemme :}
\textit{``Rand, acceptes-tu de renoncer pour toujours à tes rêves de retourner paisiblement à Emond's Field pour sauver ce village attaqué par les Trollocs ?''}
\begin{itemize}
\item Sacrifice ``Vie d'Autrefois'' (3) → +3 au test critique
\item ``Vie d'Autrefois'' passe de 3 à 2 définitivement
\end{itemize}

\subsection{Archétypes de Personnages}

\subsubsection{Canaleurs du Pouvoir Unique}

\paragraph{\textsc{Aes Sedai}}
\textit{Servantes de tous, détentrices du Pouvoir}

\textbf{Prérequis :}
\begin{itemize}
\item Origine féminine
\item Capacité de canalisation (Trait : ``Étincelle'' ou ``Apprentissage'')
\item Formation à la Tour Blanche
\end{itemize}

\textbf{Traits spécialisés :}
\begin{itemize}
\item \textbf{Serments du Bâton Liant} (3) : Incapacité magique de mentir, nuire ou créer des armes
\item \textbf{\textit{Agelessness}} (2) : Apparence intemporelle due à l'usage du Pouvoir
\item \textbf{Autorité des \textit{Aes Sedai}} (1-3) : Prestige et influence sociale
\end{itemize}

\textbf{Compétences obligatoires :}
\begin{itemize}
\item Au moins un Pouvoir (Air, Eau, Terre, Feu, Esprit) à 3+
\item Érudition 3+
\item Selon l'\textit{Ajah} : spécialisations requises
\end{itemize}

\textit{Exemple de spécialisation :} Une \textit{Aes Sedai} de l'\textit{Ajah} Jaune doit avoir Eau 5+ avec spécialisation Guérison, tandis qu'une Verte privilégiera Terre et Feu avec spécialisations militaires.

\paragraph{\textsc{Asha'man}}
\textit{Soldats du Dragon, maîtres du \textit{Saidin} souillé}

\textbf{Prérequis :}
\begin{itemize}
\item Origine masculine
\item Capacité de canalisation
\item Formation à la Tour Noire
\end{itemize}

\textbf{Traits spécialisés :}
\begin{itemize}
\item \textbf{Résistance à la Souillure} (1-3) : Retardement des effets de la folie
\item \textbf{Formation militaire} (2) : Entraînement au combat et à la guerre
\item \textbf{Instabilité latente} (1-2) : Risque de folie progressive
\end{itemize}

\textbf{Restrictions importantes :}
\begin{itemize}
\item Usage du \textit{saidin} souillé (risque de folie)
\item Suspicion générale de la population
\item Conflit interne entre factions
\end{itemize}

\paragraph{\textsc{Sagesses et Guérisseuses}}
\textit{Praticiennes traditionnelles du Pouvoir}

\textbf{Traits spécialisés :}
\begin{itemize}
\item \textbf{Savoir traditionnel} (1-3) : Connaissance des herbes et remèdes
\item \textbf{Respect villageois} (1-2) : Autorité morale dans la communauté
\item \textbf{Formation intuitive} (1-2) : Usage du Pouvoir sans formation théorique
\end{itemize}

\textbf{Limitations :}
\begin{itemize}
\item Formation limitée et empirique
\item Suspicion envers la magie ``savante''
\item Portée d'influence restreinte
\end{itemize}

\subsubsection{Guerriers et Soldats}

\paragraph{\textsc{Garde de la Reine (Andor)}}
\textit{Élite militaire andorienne}

\textbf{Traits spécialisés :}
\begin{itemize}
\item \textbf{Formation d'élite} (2-3) : Entraînement militaire supérieur
\item \textbf{Loyauté absolue} (2) : Dévouement à la Couronne d'Andor
\item \textbf{Prestance} (1) : Maintien et discipline exemplaires
\end{itemize}

\paragraph{\textsc{Enfant de la Lumière}}
\textit{Soldat de la foi contre l'Ombre}

\textbf{Traits obligatoires :}
\begin{itemize}
\item \textbf{Foi inébranlable} (2-3) : Conviction religieuse absolue
\item \textbf{Méfiance de la magie} (2) : Opposition aux utilisateurs du Pouvoir
\item \textbf{Cape blanche} (1) : Symbole d'appartenance et de pureté
\end{itemize}

\textbf{Restrictions :}
\begin{itemize}
\item Code moral strict
\item Interdiction d'utiliser la magie
\item Obligations hiérarchiques absolues
\end{itemize}

\paragraph{\textsc{Compagnon d'Illian}}
\textit{Mercenaire d'élite des Compagnies}

\textbf{Traits spécialisés :}
\begin{itemize}
\item \textbf{Vétéran des Compagnies} (1-3) : Expérience militaire variée
\item \textbf{Esprit mercenaire} (1-2) : Pragmatisme et professionnalisme
\item \textbf{Réseau de guerre} (1-2) : Contacts dans le milieu militaire
\end{itemize}

\subsubsection{Professions Civiles}

\paragraph{\textsc{Ménestrel}}
\textit{Artiste itinérant et gardien des traditions}

\textbf{Traits spécialisés :}
\begin{itemize}
\item \textbf{Voix d'or} (1-3) : Talent artistique exceptionnel
\item \textbf{Mémoire des légendes} (2) : Connaissance des histoires et traditions
\item \textbf{Bienvenu partout} (1-2) : Accueil favorable dans toutes les cultures
\end{itemize}

\textbf{Compétences favorisées :} Représentation, Folklore, Faconde, Érudition

\paragraph{\textsc{Marchand}}
\textit{Négociant voyageur entre les nations}

\textbf{Traits spécialisés :}
\begin{itemize}
\item \textbf{Œil pour les affaires} (1-3) : Intuition commerciale
\item \textbf{Réseau commercial} (1-3) : Contacts marchands étendus
\item \textbf{Connaissance des routes} (1-2) : Maîtrise des itinéraires commerciaux
\end{itemize}

\subsection{Mécaniques de Canalisation}

\subsubsection{Acquisition de la Capacité}

\textbf{L'Étincelle} : Capacité innée qui se manifeste spontanément
\begin{itemize}
\item Trait : ``Étincelle (2-3)''
\item Manifestation dangereuse sans formation
\item 1 chance sur 4 de survie sans aide
\end{itemize}

\textbf{L'Apprentissage} : Capacité latente développée par l'enseignement
\begin{itemize}
\item Trait : ``Apprentissage (1-2)''
\item Nécessite un professeur compétent
\item Progression plus lente mais plus sûre
\end{itemize}

\subsubsection{Force dans le Pouvoir}

\textbf{Niveaux de Puissance :}
\begin{itemize}
\item \textbf{Niveau 1-3} : Faible, usage domestique et personnel
\item \textbf{Niveau 4-6} : Moyen, capacités remarquables
\item \textbf{Niveau 7-9} : Fort, puissance impressionnante
\item \textbf{Niveau 10+} : Très fort, capacités légendaires
\end{itemize}

\textit{Application technique :} Le niveau de puissance détermine la quantité maximale de Pouvoir manipulable et la résistance à la fatigue magique.

\subsubsection{Spécialisations par Pouvoir}

\paragraph{\textsc{Air}}
\begin{itemize}
\item \textbf{Mouvement} : Manipulation des vents et courants aériens
\item \textbf{Son} : Contrôle des vibrations acoustiques
\item \textbf{Illusion} : Distorsion de la perception visuelle
\item \textbf{Champ de force} : Barrières et protections invisibles
\end{itemize}

\paragraph{\textsc{Eau}}
\begin{itemize}
\item \textbf{Guérison} : Réparation des blessures et maladies
\item \textbf{Purification} : Nettoyage des toxines et corruptions
\item \textbf{Glace} : Solidification et manipulation de l'eau gelée
\item \textbf{Hydratation} : Création et contrôle des liquides
\end{itemize}

\paragraph{\textsc{Terre}}
\begin{itemize}
\item \textbf{Sculpture} : Modification de la matière solide
\item \textbf{Séisme} : Manipulation tellurique et géologique
\item \textbf{Minéral} : Identification et extraction des métaux
\item \textbf{Squelette} : Guérison des fractures et malformations osseuses
\end{itemize}

\paragraph{\textsc{Feu}}
\begin{itemize}
\item \textbf{Attaque} : Boules de feu et armes enflammées
\item \textbf{Chaleur} : Contrôle de la température
\item \textbf{Combustion} : Ignition et extinction des flammes
\item \textbf{Forge} : Travail des métaux par la chaleur
\end{itemize}

\paragraph{\textsc{Esprit}}
\begin{itemize}
\item \textbf{Guérison mentale} : Soins des traumatismes psychologiques
\item \textbf{Compulsion} : Manipulation des volontés et désirs
\item \textbf{Effacement} : Suppression sélective des souvenirs
\item \textbf{Liens} : Création de connexions empathiques
\end{itemize}

\subsection{Système d'Expérience et Progression}

\subsubsection{Gains d'Expérience}

\textbf{Sources d'expérience :}
\begin{itemize}
\item \textbf{Objectifs atteints} : 1-3 points selon l'importance
\item \textbf{Roleplay} : 1 point pour incarnation réussie
\item \textbf{Découvertes} : 1-2 points pour révélations importantes
\item \textbf{Sacrifice personnel} : 1-2 points pour choix difficiles
\end{itemize}

\subsubsection{Coûts de Progression}

\textbf{Amélioration des Compétences :}
\begin{itemize}
\item \textbf{Niveau 0→1} : 1 point d'expérience
\item \textbf{Niveau 1→2} : 2 points d'expérience
\item \textbf{Niveau 2→3} : 3 points d'expérience
\item \textit{(et ainsi de suite, coût = niveau visé)}
\end{itemize}

\textbf{Nouveaux Traits :}
\begin{itemize}
\item \textbf{Trait (1)} : 2 points d'expérience
\item \textbf{Trait (2)} : 4 points d'expérience
\item \textbf{Trait (3)} : 6 points d'expérience
\end{itemize}

\textbf{Amélioration des Traits :}
\begin{itemize}
\item \textbf{Passage au niveau supérieur} : (Niveau visé × 2) points
\end{itemize}

\textit{Exemple de progression :} Un personnage voulant passer de ``Combat 3'' à ``Combat 4'' dépense 4 points d'expérience. Pour acquérir le trait ``Contacts politiques (2)'', il dépense 4 points.

\subsubsection{Spécialisations Avancées}

À partir du niveau 5 dans une Compétence, les spécialisations deviennent obligatoires pour progresser :

\textbf{Coût des Spécialisations :}
\begin{itemize}
\item \textbf{Première spécialisation} : 2 points d'expérience
\item \textbf{Spécialisations supplémentaires} : 3 points chacune
\end{itemize}

\textit{Application :} Un personnage avec ``Combat 5'' doit choisir une spécialisation (Épée, Lance, Arc, etc.) avant de pouvoir progresser vers ``Combat 6''.

\subsection{Tables de Création Rapide}

\subsubsection{Attribution Standard}

\textbf{Création de base :}
\begin{itemize}
\item \textbf{2 points de Trait} (deux Traits à 1 ou un Trait à 2)
\item \textbf{25 points de Compétences} à répartir librement
\item \textbf{Trait défavorable optionnel} pour un 3e point de Trait
\item \textbf{Traits culturels gratuits} selon l'origine (ne comptent pas dans les points)
\end{itemize}

\textbf{Exemple de répartition équilibrée :}
\begin{itemize}
\item 1 Compétence principale à niveau 6 (6 points)
\item 2 Compétences importantes à niveau 4 (8 points)
\item 4 Compétences secondaires à niveau 3 (12 points)
\item Reste distribué en compétences à 1-2
\item \textbf{Plus} 1-2 Traits culturels gratuits selon l'origine
\end{itemize}

\subsubsection{Archétypes Prêts à Jouer}

\paragraph{\textsc{Novice Aes Sedai}}
\textbf{Compétences :}
\begin{itemize}
\item Eau 4 (Guérison)
\item Érudition 4
\item Argutie 3
\item Citadin 3
\item Thaumatologie 3
\item Combat 2
\end{itemize}

\textbf{Traits :}
\begin{itemize}
\item Étincelle (2)
\item Éducation noble (1)
\item \textbf{Trait culturel gratuit} : Tradition de la Tour Blanche (1)
\end{itemize}

\paragraph{\textsc{Garde Frontalière}}
\textbf{Compétences :}
\begin{itemize}
\item Combat 5 (Épée)
\item Animaux 4 (Équitation)
\item Rural 4
\item Dépassement 3
\item Athlétisme 3
\item Citadin 2
\end{itemize}

\textbf{Traits :}
\begin{itemize}
\item Vétéran des Marches (2)
\item Vigilance constante (1)
\item \textbf{Trait culturel gratuit} : Lame des Frontières (2)
\end{itemize}

\paragraph{\textsc{Marchand Itinérant}}
\textbf{Compétences :}
\begin{itemize}
\item Faconde 4
\item Citadin 4
\item Argutie 4
\item Rural 3
\item Mer 3
\item Animaux 3
\end{itemize}

\textbf{Traits :}
\begin{itemize}
\item Réseau commercial (2)
\item Œil pour les affaires (1)
\item \textbf{Trait culturel gratuit} : Connaissance des routes (2)
\end{itemize}

\begin{quotebox}
\textbf{Note d'équilibrage :} Ces règles de création visent à produire des personnages compétents mais non surpuissants, respectant l'esprit de l'univers de Robert Jordan où même les héros commencent modestement avant de révéler leur véritable potentiel au fil des épreuves.
\end{quotebox}
