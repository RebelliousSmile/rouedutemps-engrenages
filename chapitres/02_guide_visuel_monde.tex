\section*{Chapitre 2 : Le Monde de La Roue du Temps}

\subsection{Introduction : Visualiser les Terres Connues}

Quand un voyageur pose le pied sur les routes poussiéreuses qui sillonnent les Terres de l'Ouest, il découvre un monde façonné par des millénaires d'histoire. Chaque nation porte les cicatrices et les gloires de son passé, chaque cité raconte une histoire différente, et chaque peuple a développé sa propre façon de voir le monde et de s'y adapter.

Ce chapitre ne présente pas une géographie académique, mais l'âme vivante des lieux où vos aventures prendront racine. Car pour maîtriser dans l'univers de La Roue du Temps, il faut ressentir l'atmosphère unique de chaque endroit : la tension qui règne dans les Marches, la prospérité tranquille d'Andor, l'exotisme troublant du Seanchan, ou encore la mélancolie qui baigne les ruines de l'Âge des Légendes.

Ces descriptions servent de référence immédiate pour l'improvisation en partie. Plutôt que de fouiller dans les romans à la recherche d'un détail, vous trouverez ici l'essence de chaque lieu, prête à être utilisée pour créer l'ambiance appropriée.

\begin{separateur}\end{separateur}

\subsection{Les Terres de l'Ouest}

\subsubsection{Andor - Le Royaume des Jardins}

Au cœur des Terres de l'Ouest s'étend Andor, royaume prospère où la paix règne depuis des générations. Ses vastes plaines ondulent sous les vents tièdes, ponctuées de fermes florissantes aux maisons de pierre blonde. Les routes sont bien entretenues, bordées de haies soigneusement taillées et de champs de blé doré qui s'étendent à perte de vue.

L'Andor respire la stabilité et l'ordre. Ici, les traditions ont force de loi, l'honneur familial prime sur les ambitions personnelles, et la courtoisie gouverne toutes les interactions. La Garde Royale patrouille efficacement les grands axes, et rares sont les bandits assez audacieux pour s'attaquer aux marchands sous la protection du Lion Rouge.

\begin{nationbox}{Caemlyn, la Couronne d'Andor}
La capitale s'élève en terrasses successives qui grimpent vers le ciel comme un escalier de géant. Depuis la Ville Basse aux marchés colorés jusqu'aux tours scintillantes du Palais Royal, chaque niveau est plus beau que le précédent. Les jardins ornent chaque terrasse, créant une cascade de verdure et de fleurs qui justifie la réputation de Caemlyn comme la plus belle cité des Terres de l'Ouest.

Dans la Ville Basse, les rues pavées résonnent du claquement des sabots et des conversations animées. L'air embaume les roses royales et les épices des marchés. Les auberges comme \textit{L'Épée et le Lion} accueillent voyageurs et marchands dans leurs salles communes enfumées, où l'on peut entendre les dernières nouvelles du royaume autour d'une chope de bière blonde.

La Ville Neuve étale ses demeures bourgeoises aux jardins soignés. Ici vivent les artisans prospères et les marchands établis, dans des maisons de pierre claire ornées de volets colorés. Chaque demeure rivalise avec ses voisines par la beauté de ses jardins et l'élégance de son architecture.

La Ville Interne abrite les palais nobles aux façades de marbre blanc. Les Maisons nobles d'Andor y maintiennent leurs résidences urbaines, véritables joyaux d'architecture où se déroulent bals et réceptions. Les rues y sont plus larges, les fontaines plus nombreuses, et chaque perspective soigneusement calculée pour offrir une vue harmonieuse.

Au sommet, le Palais Royal domine la ville de ses tours élancées. Ses jardins suspendus sont un miracle de l'art horticole, où fleurissent côte à côte des plantes de toutes les nations connues. C'est ici que siège la Reine-par-la-Grâce-de-La Lumière, entourée de sa cour raffinée.
\end{nationbox}

\begin{nationbox}{Les Deux Rivières - Le Refuge Oublié}
Au fin fond d'Andor, là où les cartes deviennent imprécises et où les collecteurs d'impôts ne s'aventurent plus, s'étend une région paisible que beaucoup croient disparue. Les Deux Rivières doivent leur nom au Taren et à la Manetherendrelle qui les délimitent, créant un havre de paix isolé du reste du monde.

Champ d'Emond, le village principal, sommeille au bord de la Westwood comme s'il sortait d'un conte pour enfants. Ses maisons de pierre et de bois aux toits de chaume s'alignent autour du Champ commun, où paissent paisiblement les moutons. L'Auberge de Winespring, tenue par la famille al'Vere, constitue le cœur social du village. Sa salle commune aux poutres noircies par la fumée accueille les rares voyageurs et sert de point de rassemblement pour les villageois.

La Forge de Maître Luhan résonne des coups de marteau sur l'enclume, tandis que le Moulin d'al'Dael moud le grain dans un grondement régulier. Ces bâtiments, avec la Maison commune où se réunit le Conseil du village, forment le triangle des pouvoirs locaux dans cette démocratie rurale oubliée du temps.

Les habitants des Deux Rivières cultivent un mode de vie simple mais prospère. Leurs moutons donnent une laine réputée, leurs champs de tabac produisent des feuilles recherchées, et leurs artisans créent des objets durables et beaux. Mais par-dessus tout, ils chérissent leur tranquillité et leur autonomie, ignorant superbement les troubles du vaste monde.
\end{nationbox}

\subsubsection{Cairhien - L'Échiquier Politique}

Au sud d'Andor s'étend Cairhien, royaume où la politique est un art de vivre et l'intrigue une seconde nature. Ses paysages vallonnés et ses forêts sombres abritent un peuple petit de taille mais grand en subtilité, où chaque mot peut cacher trois significations et chaque geste porter un message.

La nation entière semble prise dans une partie d'échecs géante où nobles et bourgeois manœuvrent sans relâche pour améliorer leur position. Le Grand Jeu, comme on appelle cette politesse ritualisée qui cache des luttes de pouvoir impitoyables, imprègne tous les aspects de la société cairhienaise.

\begin{nationbox}{La Cité de Cairhien}
La capitale elle-même reflète cette complexité. Ses rues s'entrecroisent selon un plan géométrique strict, créant des perspectives calculées et des angles étudiés. Les tours de pierre grise s'élancent vers le ciel comme des flèches, chacune tentant de dominer ses voisines. Mais cette apparente harmonie cache une réalité plus sombre : chaque bâtiment, chaque rue, chaque fontaine peut servir de pion dans le Grand Jeu.

Le Soleil Levant, palais royal aux tours jumelles, domine la ville de sa masse imposante. C'est ici que le Trône du Soleil Levant arbitre les destinées du royaume, entouré d'une cour où l'étiquette atteint des sommets de complexité. Un regard mal dirigé, un silence trop long, un sourire déplacé peuvent déclencher des conséquences qui se répercuteront pendant des générations.

Dans les salons nobles, les dames rivalisent d'élégance et de finesse, leurs robes sombres aux décolletés précis servant d'armes dans leurs duels de charme et d'intelligence. Les seigneurs, avec leurs justaucorps ornés et leurs épées d'apparat, excellent dans l'art du sous-entendu et de la menace voilée.
\end{nationbox}

\subsubsection{Le Shienar - Rempart Contre L'Ombre}

Aux frontières de la Flétrissure, le Shienar dresse ses forteresses comme un défi lancé aux Ténèbres. Cette nation de guerriers a fait de la lutte contre L'Ombre sa raison d'être, transformant chaque homme en soldat et chaque femme en soutien indéfectible de cette guerre éternelle.

\begin{nationbox}{Fal Dara - La Citadelle de Granite}
La forteresse de Fal Dara incarne l'esprit du Shienar. Ses murailles de granite noir s'élèvent comme une montagne artificielle, percées de meurtrières et couronnées de tours de guet d'où les sentinelles scrutent sans relâche la Flétrissure. Nulle ornementation ne vient adoucir cette architecture militaire : chaque pierre a été taillée pour la guerre, chaque angle calculé pour la défense.

Dans les cours d'entraînement résonne sans cesse le choc des épées et les cris des instructeurs. Les soldats du Shienar s'exercent du lever au coucher du soleil, perfectionnant les techniques qui leur permettront de survivre face aux Trollocs et aux Myrddraal. Leur discipline égale celle des \textit{Aes Sedai}, leur courage celui des héros légendaires.

Mais Fal Dara n'est pas qu'une caserne. Dans ses halls éclairés par de grands braseros, l'hospitalité du Shienar s'épanouit avec une générosité qui surprend le voyageur. Car si ces gens vivent pour la guerre, ils savent aussi que chaque jour de paix est un trésor à chérir, chaque ami un cadeau précieux dans ce monde dur.
\end{nationbox}

\subsubsection{Tear - L'Orgueil des Seigneurs}

Au sud-est des Terres de l'Ouest, Tear étale sa richesse et son arrogance avec une ostentation qui défie la prudence. Cette nation de marchands et de nobles vit du commerce qui remonte l'Erinin depuis la Mer des Tempêtes, transformant sa capitale en l'un des ports les plus riches du monde connu.

\begin{nationbox}{La Cité de Tear}
La ville elle-même témoigne de cette prospérité insolente. Ses quais de pierre s'étendent sur des lieues le long du fleuve, encombrés de navires aux voiles colorées venus de tous les coins du monde. Les entrepôts débordent de marchandises exotiques : épices de Mayene, soies de Tarabon, métaux précieux des Montagnes de la Brume.

Dominant le port, la Pierre de Tear dresse sa masse colossale comme un défi au ciel lui-même. Cette forteresse-palais, taillée dans un seul bloc de pierre noire, abrite les Hauts Seigneurs et leurs trésors incommensurables. Ses murs sans fenêtres aux niveaux inférieurs racontent l'histoire d'un peuple qui a toujours vécu dans la crainte de l'invasion, mais dont la richesse a finalement vaincu tous les ennemis.

Dans les rues de la ville haute, les palais nobles rivalisent d'extravagance. Leurs façades de marbre polychrome, leurs jardins suspendus et leurs fontaines d'or liquide témoignent d'une richesse si grande qu'elle en devient presque obscène. Car à Tear, afficher sa fortune n'est pas de la vanité, c'est une nécessité politique.
\end{nationbox}

\begin{separateur}\end{separateur}

\subsection{Les Peuples des Terres Connues}

\subsubsection{Les Aiels - Guerriers du Désert}

À l'est de l'Épine du Monde, dans le désert impitoyable des Terres Perdues, vit un peuple qui défie toutes les conventions des Terres de l'Ouest. Les Aiels ont fait de la guerre un art, de l'honneur une religion, et de la survie dans des conditions impossibles leur fierté suprême.

Grands et décharnés, le visage hâlé par le soleil du désert, les guerriers aiels se déplacent avec la grâce mortelle des prédateurs. Leurs voiles colorés et leurs vêtements de cuir souple leur permettent de disparaître dans le paysage désertique comme s'ils faisaient partie des rochers et du sable. Armés de leurs lances courtes et de leurs boucliers de cuir noir, ils peuvent couvrir des distances prodigieuses en courant, apparaissant là où l'ennemi les attend le moins.

Mais les Aiels ne sont pas que des guerriers. Leur société complexe s'articule autour du \textit{ji'e'toh}, un code d'honneur si subtil et si contraignant qu'il régit chaque aspect de leur existence. Pour un Aiel, mourir sans honneur est pire que ne jamais naître, et cette conviction transforme même le plus humble d'entre eux en héros potentiel.

Les femmes aiels, Vierges de la Lance ou \textit{Femmes Sagettes}, égalent les hommes en courage et en compétence. Les Vierges combattent aux côtés des hommes dans la \textit{algai'd'siswai}, la danse des lances, tandis que les \textit{Femmes Sagettes} manient des pouvoirs mystérieux qui font d'elles les arbitres ultimes des disputes entre clans.

\subsubsection{Les Seanchans - L'Empire aux Mille Visages}

De l'autre côté de l'Océan d'Aryth est venu un peuple qui a bouleversé l'équilibre des Terres de l'Ouest par sa seule apparition. L'Empire Seanchan apporte avec lui des coutumes si étranges, une organisation si rigide et des créatures si extraordinaires qu'il semble sortir des légendes les plus folles.

L'ordre seanchan repose sur une hiérarchie immuable où chacun connaît sa place exacte dans la société. Du plus humble fermier jusqu'à l'Impératrice elle-même, que l'on puisse vivre à jamais, chaque individu porte ses marques de rang avec une fierté mêlée de résignation. Car dans l'Empire, contester l'ordre établi n'est pas seulement un crime, c'est une impossibilité conceptuelle.

Les nobles seanchans, avec leurs crânes rasés ornés de tatouages complexes et leurs ongles laqués de couleurs codifiées, évoluent dans un monde de protocole rigide où chaque geste a sa signification. Leurs robes aux motifs élaborés, leurs masques d'apparat et leurs éventails parlent un langage que seuls les initiés comprennent.

Mais c'est dans leur rapport à la canalisation que les Seanchans révèlent leur vraie nature. Les \textit{damane}, femmes capables de toucher Le Pouvoir Unique, sont enchaînées et traitées comme des animaux de compagnie par leurs \textit{sul'dam}. Cette pratique, qui horrifie les \textit{Aes Sedai}, semble naturelle aux Seanchans : pour eux, laisser une canalisatrice libre revient à lâcher un fauve enragé dans une foule.

\subsubsection{Les Gens de la Mer - Maîtres des Océans}

Sur les vagues houleuses qui séparent les continents naviguent les Gens de la Mer, peuple nomade qui a fait des océans sa patrie et des navires ses cités. Leurs flottes aux voiles triangulaires sillonnent toutes les mers connues, apportant nouvelles, marchandises et merveilles dans chaque port qu'ils visitent.

Les Gens de la Mer ont développé une culture unique, mélange d'indépendance farouche et de discipline collective. Chaque navire constitue une communauté autonome dirigée par sa \textit{Maîtresse de Voile}, mais l'ensemble de la flotte obéit à la \textit{Maîtresse des Vagues}, autorité suprême reconnue par tous les clans navigateurs.

Leurs tatouages colorés racontent l'histoire de leurs voyages et de leurs exploits. Plus un Homme de la Mer ou une Femme de la Mer porte de marques, plus son statut est élevé et son expérience respectée. Ces dessins complexes couvrent souvent tout le corps, transformant chaque individu en livre vivant de l'histoire maritime.

\begin{separateur}\end{separateur}

\subsection{Les Lieux de Légende}

\subsubsection{Tar Valon - L'Île de la Tour Blanche}

Au cœur de l'Erinin s'élève l'île de Tar Valon, miracle d'architecture et centre spirituel des Terres de l'Ouest. La cité entière semble sculptée dans l'ivoire et le marbre, ses tours gracieuses et ses ponts élégants défiant les lois de l'équilibre avec une beauté qui coupe le souffle.

La Tour Blanche domine cette merveille, s'élançant vers les nuages comme une lance de lumière pure. Dans ses étages innombrables vivent et étudient les \textit{Aes Sedai}, femmes de pouvoir dont l'influence se fait sentir dans tous les royaumes. Chaque étage de la Tour raconte une partie de l'histoire du monde, depuis les archives profondes jusqu'aux appartements de l'Amyrlin au sommet.

Mais Tar Valon n'est pas que la Tour. Ses jardins suspendus rivalisent avec ceux de Caemlyn, ses bibliothèques contiennent les sagesses de tous les peuples, et ses écoles forment les érudits de demain. C'est une cité où le savoir prime sur la richesse, où l'intelligence compte plus que la force, et où l'avenir du monde se décide souvent autour d'une simple tasse de thé.

\subsubsection{Rhuidean - La Cité Perdue des Aiels}

Cachée au cœur du désert aiel se dresse Rhuidean, cité fantôme qui garde les secrets d'un peuple. Interdite à tous sauf aux chefs de clan et aux \textit{Femmes Sagettes} en devenir, elle recèle les vestiges d'un passé que la plupart des Aiels préféreraient oublier.

Ses rues de verre et de \textit{ter'angreal} brillent sous le soleil du désert comme un mirage fait réalité. Chaque bâtiment raconte une histoire différente de l'Âge des Légendes, époque où les ancêtres des Aiels servaient la paix plutôt que la guerre. Pour ceux qui y pénètrent, Rhuidean offre la vérité sur leurs origines, mais cette vérité est souvent plus difficile à porter que l'ignorance.

Les colonnes de cristal qui s'élèvent au centre de la cité montrent à ceux qui les touchent l'histoire complète de leur lignée, remontant jusqu'à l'Âge des Légendes. Cette révélation brise certains hommes, en élève d'autres au rang de chef, mais ne laisse jamais quiconque inchangé.

\subsubsection{Les Terres Dévastées - Cicatrices du Monde}

Au nord du Shienar s'étend la Flétrissure, terre maudite où la géographie elle-même porte les stigmates de la folie de Lews Therin Telamon. Ici, les lois naturelles ne fonctionnent plus correctement : l'été peut céder la place à l'hiver en quelques pas, les rivières coulent vers le haut, et les collines changent de place d'un jour à l'autre.

Mais au cœur de cette désolation se trouve un lieu plus terrible encore : Shayol Ghul, la Montagne du Destin où fut percée la prison du Ténébreux. De cette blessure dans le tissu même de la réalité s'échappent encore les miasmes de L'Ombre, corrompant tout ce qu'ils touchent et donnant naissance aux créatures les plus hideuses que le monde ait connues.

Les rares voyageurs qui survivent à un passage dans les Terres Dévastées en reviennent changés, marqués par des visions de cauchemar et une compréhension trop claire de ce que représente vraiment la lutte entre Lumière et Ténèbres. Car ici, le mal n'est pas une abstraction philosophique, mais une force tangible qui ronge l'âme et corrompt l'esprit.

\begin{separateur}\end{separateur}

\subsection{Conclusion : Vivre le Monde}

Ces descriptions ne constituent qu'un aperçu de la richesse infinie de l'univers de La Roue du Temps. Chaque lieu possède ses propres mystères, chaque peuple ses propres traditions, et chaque cité ses propres secrets. L'important n'est pas de mémoriser chaque détail, mais de saisir l'essence de chaque endroit pour pouvoir la transmettre à vos joueurs.

Car dans une partie de jeu de rôle, ce ne sont pas les statistiques qui créent l'immersion, mais l'émotion. Savoir que Caemlyn sent les roses et résonne des conversations polies vaut mieux que connaître le nombre exact de ses habitants. Comprendre que les Aiels vivent et meurent pour l'honneur est plus utile que retenir par cœur tous les préceptes du \textit{ji'e'toh}.

Utilisez ces descriptions comme des tremplins pour votre imagination. Adaptez-les aux besoins de votre histoire, enrichissez-les de vos propres détails, et surtout, donnez-leur vie à travers les yeux de vos personnages. Car c'est ainsi que le monde de La Roue du Temps prendra véritablement forme autour de votre table de jeu.
